\documentclass[11pt]{beamer}
\usetheme[bullet=circle,% Use circles instead of squares for bullets.
          titleline=true,% Show a line below the frame title.
          alternativetitlepage=true,% Use the fancy title page.
          ]{Torino}

\usepackage{color}
\usepackage[utf8]{inputenc}
\usepackage[T1]{fontenc}
\usepackage[ngerman,english]{babel}
\usepackage{url}
\usepackage{tabularx}
\usepackage{booktabs}
\usepackage{subfig}
\usepackage{caption}
\newcommand{\tableheadline}[1]{#1}

\usecolortheme{skyblue}

\graphicspath{{../doc/figures/}}
\captionsetup{format=hang,font=small}

\newlength{\colwidth}
\setlength{\colwidth}{0.465\textwidth}

\author{Daniel Siegel}
\title{Typical Development Processes of}% Free and Open Source Software Projects}
\subtitle{Free and Open Source Software Projects}
\institute{Technische Universität München}
\date{8. Mai 2012}

\begin{document}
\begin{frame}[t,plain]
  \titlepage
\end{frame}

\begin{frame}[t]{Inhalt}
  \begin{itemize}
    \item Motivation \& Problemstellung
    \item Kriterienkatalog \& Projektauswahl
    \item Methodik
    \item Projektanalyse
    \item Evaluation \& Ausblick
  \end{itemize}
\end{frame}

\begin{frame}{Problemstellung}

\begin{beamercolorbox}[sep=1em]{title page header}
  Wie funktionieren Freie und Open Source Software (FOSS) Projekte, welche
  Strukturen haben diese und welche Arbeitsweisen werden verwendet. Dabei
  werden FOSS Projekte dahingehend untersucht, gemeinsame Modelle zu finden und
  diese in einer Klassifizierung darzustellen. Zusätzlich werden diese mit
  traditionellen Softwareentwicklungsmodellen verglichen um Gemeinsamkeiten
  der Unterschiede aufzuzeigen.
\end{beamercolorbox}
\end{frame}

\begin{frame}{Kriterien für die Projektauswahl}
  \begin{beamercolorbox}[sep=1em]{title page header}
    Ziel: Ermöglichung einer guten Auswahl der Projekte, in der eine Analyse
    sinnvolle Ergebnisse liefert.
  \end{beamercolorbox}

  \begin{columns}[T]
  \column{\colwidth}
  \begin{itemize}
    \item Popularität
    \item Alter
    \item Kategorie
    \item Aktivität
    \begin{itemize}
      \item Releases
      \item Downloads
      \item Commits
    \end{itemize}
  \end{itemize}
  \column{\colwidth}
  \begin{itemize}
    \item Community
    \begin{itemize}
      \item Kommunikation
      \item Anzahl der Entwickler
      \item Konferenzen
      \item Foundations
      \item Laufende Projekte
    \end{itemize}
  \end{itemize}
  \end{columns}
\end{frame}

\begin{frame}[t]{Liste der analysierten Projekte}
  \begin{center}
  \begin{tabularx}{\textwidth}{llX}
    \toprule
    \tableheadline{Project} & \tableheadline{Age} & \tableheadline{Category} \\
    \midrule
    Debian        & 1993 & Operating System \\
    Drupal        & 2001 & Content Management System \\
    Fedora        & 2002 & Operating System \\
    GNOME         & 1997 & Desktop Environment \\
    KDE           & 1996 & Desktop Environment \\
    MySQL/MariaDB & 1997 & Database Management System \\
    PHP           & 1994 & Interpreted Programming Language \\
    Plone         & 1999 & Content Management System \\
    PostgreSQL    & 1986 & Database Management System \\
    Python        & 1989 & Interpreted Programming Language \\
    \bottomrule
  \end{tabularx}

  \end{center}
\end{frame}

\begin{frame}[t]{Grounded Theory}
  \begin{columns}[T]
  \column{\colwidth}
    \begin{itemize}
      \item Glaser und Strauss, 1967
      \item Sozialwissenschaftlicher Ansatz zur systematischen Auswertung von qualitativen Daten
      \item Ziel: Theoriegenerierung
      \item Differenzen zwischen Glaser und Strauss
      \begin{itemize}
          \item Seit den 1970ern
          \item Wichtigster Unterschied: Kodierung nach Kategorien
      \end{itemize}
    \end{itemize}
  \column{\colwidth}
    \includegraphics[height=0.74\textheight]{grounded_theory}

    {\tiny\hfill
    Strauss and Corbin (1996)
    }
  \end{columns}
\end{frame}

\begin{frame}[t]{Qualitative Inhaltsanalyse nach Mayring}
  \begin{columns}[T]
  \column{\colwidth}
    \begin{itemize}
      \item Mayring, 1983
      \item Ansatz für eine empirische, aber methodisch kontrollierte Analyse
      \item Ziel: Erstellung einer systematischen und intersubjektiv überprüfbaren Analysemethode
      \item Kernstück: Festlegung der Kategorien und ihrer Abstraktion
    \end{itemize}
  \column{\colwidth}
    \includegraphics[height=0.74\textheight]{mayring}

    {\tiny\hfill
    Mayring (1983, 2000)
    }
  \end{columns}
\end{frame}

\begin{frame}{Analyse: Beispiel aus Python}
  \begin{center}
    \includegraphics[height=0.8\textheight]{python/commits_by_author}
  \end{center}
\end{frame}

\begin{frame}{Analyse: Beispiel aus KDE}
  \begin{center}
    \includegraphics[height=0.8\textheight]{kde/commits_by_year}
  \end{center}
\end{frame}

\begin{frame}{Analyse: Beispiel aus GNOME}
  \begin{center}
    \includegraphics[width=\textwidth]{gnome/punchcard}
  \end{center}
\end{frame}

\begin{frame}{Analyse: Beispiel aus Drupal}
  \begin{center}
    \includegraphics[height=0.8\textheight]{drupal/commits_by_month}
  \end{center}
\end{frame}

\begin{frame}{Analyse: Beispiel aus PHP}
  \begin{center}
    \includegraphics[height=0.8\textheight]{php/authors_by_month}
  \end{center}
\end{frame}

\begin{frame}{Project Analysis Catalogue}
  \begin{columns}[T]
  \column{\colwidth}
  \begin{enumerate}
    \item Description of the Project
    \item Project Category
    \item Scope of Analysis
    \item License
    \item History
    \begin{enumerate}
      \item Founders
      \item Project Age
    \end{enumerate}
    \item Community
    \begin{enumerate}
      \item Community Size
      \item Communication
      \item Conferences and Meet-Ups
      \item Roles
      \item Role of the Founders
    \end{enumerate}
  \end{enumerate}
  \column{\colwidth}
  \begin{enumerate}
    \setcounter{enumi}{6}
    \item Release Process
    \begin{enumerate}
      \item Version Naming
      \item Characterization of Releases
      \item Release Schedule
      \item Important Steps in the Schedule
    \end{enumerate}
    \item Development
    \begin{enumerate}
      \item Development Lead
      \item Development Workflow
      \item Feature Inclusion Process
    \end{enumerate}
  \end{enumerate}
  \end{columns}
\end{frame}

\begin{frame}[T]{Resultate}
  \begin{itemize}
    \item Validierung des Projekt-Kataloges
    \item Clusterbildung
    \begin{itemize}
      \item Geschichte und Evolution
      \item Struktur
      \item Veröffentlichungsprozess
      \item Entwicklungsprozess \& -modelle
    \end{itemize}
  \end{itemize}
\end{frame}

\begin{frame}{Geschichte}
  \begin{columns}
  \column{\colwidth}
  \begin{itemize}
    \item Unterschiedliche Herkunft
    \begin{itemize}
      \item Forschung
      \item Persönliche Motive
      \item Kommerzielle Motive
      \item Philosophische Motive
    \end{itemize}
    \item Kleine Gruppe von Gründern
    \item Wachstumsschub nach Erst-Veröffentlichung
    \item Weitere Wachstumsschübe vor großen Releases
  \end{itemize}
  \column{\colwidth}
    \includegraphics[width=\textwidth]{gnome/authors_by_month}

    {\tiny\hfill
    GNOME
    }
  \end{columns}
\end{frame}

\begin{frame}{Struktur (1)}
\begin{figure}[htbp]
  \centering
  \setcounter{subfigure}{0}
  \subfloat[Drupal]
    {\includegraphics[width=.17\textwidth]{drupal/roles}} \quad
  \subfloat[Plone]
    {\includegraphics[width=.17\textwidth]{plone/roles}} \quad
  \subfloat[Python]
    {\includegraphics[width=.17\textwidth]{python/roles}} \quad
  \subfloat[PHP]
    {\includegraphics[width=.17\textwidth]{php/roles}} \quad
  \subfloat[GNOME]
    {\includegraphics[width=.17\textwidth]{gnome/roles}} \\

  \vfill

  \subfloat[KDE]
    {\includegraphics[width=.17\textwidth]{kde/roles}} \quad
  \subfloat[PostgreSQL]
    {\includegraphics[width=.17\textwidth]{postgresql/roles}} \quad
  \subfloat[MariaDB]
    {\includegraphics[width=.17\textwidth]{mariadb/roles}} \quad
  \subfloat[Fedora]
    {\includegraphics[width=.17\textwidth]{fedora/roles}} \quad
  \subfloat[Debian]
    {\includegraphics[width=.17\textwidth]{debian/roles}} \\
\end{figure}
\end{frame}

\begin{frame}{Struktur (2)}
  \begin{center}
    \includegraphics[height=0.74\textheight]{structure}

    {\tiny\hfill
    Crowston et al. (2005)
    }
  \end{center}
\end{frame}

\begin{frame}{Veröffentlichungsprozess}
  \vspace{-1.4em}
\begin{figure}[htbp]
  \centering
  \setcounter{subfigure}{0}
  \subfloat[PHP]
    {\includegraphics[width=.46\textwidth]{php/cycle}}
  \subfloat[GNOME]
    {\includegraphics[width=.46\textwidth]{gnome/cycle}} \\
  \subfloat[KDE]
    {\includegraphics[width=.46\textwidth]{kde/cycle}}
  \subfloat[Fedora]
    {\includegraphics[width=.46\textwidth]{fedora/cycle}}
\end{figure}
\end{frame}

\begin{frame}{Entwicklungsprozess}
  \includegraphics[width=\textwidth]{gnome/feature}
\end{frame}

\begin{frame}{Entwicklungsmodelle: Vergleich Wasserfallmodell}
  \begin{center}
  \begin{columns}[T]
  \column{\colwidth}
  \begin{itemize}
    \item Royce, 1970
    \item Oft zu finden in
    \begin{itemize}
      \item Jungen Projekten
      \item Protoypen
      \item Module
    \end{itemize}
    \item Wiederverwendung der einzelnen Phasen
    \item Iterative Modelle herrschen jedoch vor
  \end{itemize}
  \column{\colwidth}
  \includegraphics[width=\textwidth]{waterfall}

  {\tiny\hfill
  Royce (1970)
  }

  \end{columns}
  \end{center}
\end{frame}

\begin{frame}{Entwicklungsmodelle: Vergleich Spiralmodell}
  \begin{center}
  \begin{columns}[T]
  \column{\colwidth}
  \begin{itemize}
    \item Boehm, 1988
    \item Iteratives und evolutionäres Modell
    \item Jedoch fixe Einteilung der einzelnen Phasen
    \item Teilweise fehlende Re-Evaluation
    \item Teilweise komplexere Modelle
  \end{itemize}
  \column{\colwidth}
  \includegraphics[width=\textwidth]{spiral}

  {\tiny\hfill
  Boehm (1988)
  }

  \end{columns}
  \end{center}
\end{frame}

\begin{frame}{Entwicklungsmodelle: Vergleich Agile Methoden}
  \begin{center}
  \begin{columns}[T]
  \column{0.33\textwidth}
  \begin{itemize}
    \item Hohe Ähnlichkeit
    \begin{itemize}
      \item Agile Prozesse
      \item Enge Beziehung mit Benutzern
      \item Schnelle Entwicklung und Iteration
      \item Prototypen
    \end{itemize}
    \item Allerdings
    \begin{itemize}
      \item Keine Co-Lokation
      \item \emph{Führungskräfte}
      \item Kundeninteraktion
    \end{itemize}
  \end{itemize}
  \column{0.6\textwidth}
  \includegraphics[width=\textwidth]{xp}

  {\tiny\hfill
  Extreme Programming, Beck (1999)
  }
  \vspace{1em}

  \includegraphics[width=\textwidth]{scrum}

  {\tiny\hfill
  Scrum, Sutherland and Schwaber (1995)
  }
  \end{columns}
  \end{center}
\end{frame}

\begin{frame}{Entwicklungsmodell für FOSS Projekte}
  \begin{center}
    \includegraphics[width=\textwidth]{ossd}

    {\tiny\hfill
    Roets, Minnaar and Wright (2007)
    }
  \end{center}
\end{frame}


\begin{frame}{Zusammenfassung}
  \begin{itemize}
    \item Viele Gemeinsamkeiten
    \begin{itemize}
    \item Hierachie
    \item Ähnliche Veröffentlichungsprozesse
    \item Keine Deckung mit traditionellen oder agilen SE Methoden
    \item Gemeinsames Entwicklungsmodell für FOSS Projekte
    \end{itemize}
    \item Ausblick
    \begin{itemize}
      \item Zusätzliche Projekte
      \item Evolution der Entwicklungsmodelle
      \item Fixierte Releasezyklen
    \end{itemize}
  \end{itemize}
\end{frame}

\begin{frame}
  \begin{center}
  {\Large Vielen Dank für die Aufmerksamkeit}
  \end{center}
\end{frame}

\end{document}
