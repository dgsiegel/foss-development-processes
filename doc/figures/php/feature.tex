\documentclass{standalone}

\usepackage{mathpazo}
\usepackage{tikz}
\usepackage{../../styles/tangocolors}

\usetikzlibrary{calc}
\usetikzlibrary{positioning}
\usetikzlibrary{shapes}

\begin{document}

\tikzstyle{box} = [rectangle, rounded corners, draw=black, text width=7em, minimum height=3em, text centered]

\begin{tikzpicture}
 \node[box]
   (proposal) {Feature Proposal};
 \node[minimum height=3em, below=0.5cm of proposal]
   (dummy) {};
 \node[box, below=0.5cm of dummy]
   (withdrawn) {Withdrawn};

 \node[box, right=1cm of proposal]
   (discussion) {Discussion};
 \node[box, below=0.5cm of discussion]
   (decision) {Voting};
 \node[box, below=0.5cm of decision]
   (rejected) {Rejected};

 \node[box, right=1cm of discussion]
   (implementation) {Implementation};
 \node[minimum height=3em, below=0.5cm of implementation]
   (review) {};
 \node[box, below=0.5cm of review]
   (accepted) {Accepted};

 \draw[draw, -latex] (proposal.south) -- (withdrawn.north);
 \draw[draw, -latex] (proposal.east) -- (discussion.west);

 \draw[draw, -latex] (discussion.south) -- (decision.north);
 \draw[draw, -latex] (decision.south) -- (rejected.north);
 \path[draw, -latex] (decision.east) -| ($(decision.east) + (0.5cm,0)$) |- (implementation.west);

 \draw[draw, -latex] (implementation.south) -- (accepted.north);

\end{tikzpicture}

\end{document}

