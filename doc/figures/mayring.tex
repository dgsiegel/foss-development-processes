\documentclass{standalone}

\usepackage{mathpazo}
\usepackage{tikz}
\usepackage{../styles/tangocolors}

\usetikzlibrary{calc}
\usetikzlibrary{positioning}
\usetikzlibrary{shapes}

\begin{document}

\tikzstyle{box} = [rectangle, rounded corners, draw=black, text width=20em, minimum height=3em, text centered]

\begin{tikzpicture}
 \node[box]
   (research) {Research Question};
 \node[box, below=2em of research]
   (category) {Determination of Category Definition and Levels of Abstraction};
 \node[box, below=2em of category]
   (category2) {Step by Step Formulation and Subsumption of Inductive Categories};
 \node[box, below left=2em and -3.59cm of category2, text width=9.5em]
   (revision) {Revision of Categories};
 \node[box, right=1.3em of revision, text width=8.5em]
   (formative) {Formative Check of Reliability};
 \node[box, below left=7em and -3.59cm of category2, text width=9.5em]
   (final) {Final Working through the Content};
 \node[box, right=1.3em of final, text width=8.5em]
   (summative) {Summative Check of Reliability};
 \node[box, below=12em of category2]
   (interpretation) {Interpretation of Results};

 \draw[draw, -latex] (research.south) -- (category.north);
 \draw[draw, -latex] (category.south) -- (category2.north);
 \draw[draw, -latex] (category2.south) -- (revision.north);
 \draw[draw, -latex] (revision.south) -- (final.north);
 \draw[draw, -latex] (final.south) -- (interpretation.north);

 \draw[draw, -latex] (revision.east) -- (formative.west);
 \draw[draw, -latex] (final.east) -- (summative.west);

 \path[draw, -latex] (formative.east) -| +(right:1em) |- (research.east);
 \path[draw, -latex] (formative.east) -| +(right:1em) |- (category.east);
 \path[draw, -latex] (interpretation.west) -| +(left:1em) |- (research.west);

\end{tikzpicture}

\end{document}

