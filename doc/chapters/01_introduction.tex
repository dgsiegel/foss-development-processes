\chapter{Introduction} % {{{
\label{chap:introduction}

\section{Problem and Motivation} % {{{

% }}}

\section{Outline of the Thesis} % {{{

\paragraph{\autoref{chap:introduction} -- \nameref{chap:introduction}}

This chapter presents an overview of the thesis and introduces the reader to
the topic of \ac{FOSS}.

\paragraph{\autoref{chap:theoretical background} -- \nameref{chap:theoretical background}}

In order to use a scientific research method for the project analysis, several
research methods and tools will be introduced in this chapter. Additionally
traditional and agile software engineering methods will be discussed.

\paragraph{\autoref{chap:related work} -- \nameref{chap:related work}}

An outline of related researches in the field of \ac{FOSS}, the
development processes, case studies and software engineering methods.

\paragraph{\autoref{chap:methodology} -- \nameref{chap:methodology}}

The used methodology and explanations of visualized project data will be
provided in this chapter.

\paragraph{\autoref{chap:analysis} -- \nameref{chap:analysis}}

The study takes a deeper look at several \ac{FOSS} projects, coming up with an
analysis catalogue and applying that on further projects.

\paragraph{\autoref{chap:comparison} -- \nameref{chap:comparison}}

This chapter examines the previous made analysis and compares the found findings.

\paragraph{\autoref{chap:discussion} -- \nameref{chap:discussion}}

This chapter will evaluate and analyze the findings and compare them with
traditional software engineering methods.

\paragraph{\autoref{chap:conclusion} -- \nameref{chap:conclusion}}

The thesis finally concludes and draws together the main findings of the study.

% }}}

% }}}
