\chapter{Introduction} % {{{
\label{chap:introduction}

\section{Problem and Motivation} % {{{

Without \ac{FOSS} projects the modern daily grind would be unthinkable. They
empower \emph{Fortune 500} companies such as Google, Amazon, Red Hat or
Facebook. They also are a key figure in the internet, running on most of the
servers, but also on smart phones or other devices. However \ac{FOSS} projects
are not only being used, they are also developed actively. The development is
not only done by volunteers but also by companies.

Lowdown such as the costs, the free availability of the source code and the
number of persons who work on a single project are certainly positive aspects
of \ac{FOSS} projects. They allow to use them without big investments and risks
to implement an idea quickly and effective. A matching example is the company
Google which used the Linux operating system to build a big data center
cost-effectively. That is just one of the many reasons why \ac{FOSS} is quite
popular especially in the last years.

On account of the different origins and backgrounds of the single projects,
different development models evolved, which look inimitable. Examples are the
different communities, development and release cycles, the used tools or the
project structure. When comparing them at first glance they indeed look quite
different, too diverse seem the different project leaders, the communities and
the goals of each project. The question here is now whether projects have
common grounds or use so called \emph{best practices}.

Unfortunately there are not too many scrutinies which aim for similarities or
differences in the development processes of \ac{FOSS} projects. There exist
however many studies about single aspects in \ac{FOSS} projects, such as the
motivation of developers, the social structure inside a project, the
communication, development workflows or software engineering methods. This thesis
should provide a primary step into this direction, summarizing the findings and
trying to establish a common ground.

Concretely, this study wants to answer the following research
question: How do \ac{FOSS} projects work, which structures do they have and
which workflows have they established. To accomplish this, several \ac{FOSS}
will be analyzed in order to find concertedly models. In addition they will be
compared to traditional software engineering models.

% }}}

\section{Outline of the Thesis} % {{{

\paragraph{\autoref{chap:introduction} -- \nameref{chap:introduction}}

This chapter presents an overview of the thesis and introduces the reader to
the problem and motivation of this analysis.

\paragraph{\autoref{chap:theoretical background} -- \nameref{chap:theoretical background}}

In order to use a scientific research method for the project analysis, several
research methods and tools will be introduced in this chapter. Additionally
traditional and agile software engineering methods will be discussed.

\paragraph{\autoref{chap:related work} -- \nameref{chap:related work}}

An outline of related researches in the field of \ac{FOSS}, the development
processes, case studies and software engineering methods is presented with the
appropriate references.

\paragraph{\autoref{chap:methodology} -- \nameref{chap:methodology}}

The used methodology and explanations of visualized project data will be
provided in this chapter. This includes a general explanation of methods and
the presentation of collected data.

\paragraph{\autoref{chap:analysis} -- \nameref{chap:analysis}}

A deeper look is made at several \ac{FOSS} projects, coming up with an analysis
catalogue and applying that on further projects. This analysis will be the
center piece on which further chapters build on.

\paragraph{\autoref{chap:comparison} -- \nameref{chap:comparison}}

This chapter examines the previous made analysis and compares the found
findings by working through the previously established catalogue and analyses.

\paragraph{\autoref{chap:discussion} -- \nameref{chap:discussion}}

The findings will be evaluated, analyzed and compared with traditional software
engineering methods, related findings by other researchers and previously
explained methods.

\paragraph{\autoref{chap:conclusion} -- \nameref{chap:conclusion}}

The thesis finally concludes with a summary and draws together the main
findings of the study along with possible future directions.

% }}}

% }}}
