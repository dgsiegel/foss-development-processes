\chapter{Conclusion} % {{{
\label{chap:conclusion}

This thesis started from the problem and motivation of development processes in
\ac{FOSS} projects. It defined the main goal of the work, which was to analyze
the development workflows of \ac{FOSS} projects and to finally find matching
patterns.

Next, the used theoretical background was explained and established. This
included research methods such as the Grounded Theory or the Qualitative
Content Analysis. The usage of such methods was vindicated by similar research
papers and studies which also use same or similar research methods. Also
traditional and agile software engineering methods were offered for later use.
As traditional software engineering methods the Waterfall and Spiral model were
explained. The instances of agile software engineering methods were the
\acl{XP} and Scrum software engineering models.

The methodology of how projects were chosen was defined. Based on that,
\ac{FOSS} projects were analyzed and categorized to find matching projects. The
\ac{FOSS} projects Debian, Drupal, Fedora, GNOME, KDE, MySQL/MariaDB, PHP,
Plone, PostgreSQL and Python were chosen. Next, automated gathering of project
data and the visualization thereof was discussed and explained. This determined
the further analysis and the constitution of the work.

Data about the projects was collected and the chosen projects were analyzed in
depth using the Grounded Theory research method. Six projects were examined
thoroughly to identify a project analysis catalogue. Until this point the
Grounded Theory approach was used. The catalogue was then applied to the other
four projects with Qualitative Content Analysis, as proposed earlier.

Finally the outcome of the previous project analysis was compared. According to
the project analysis catalogue differences and similarities between the
projects were distinguished. This included all development related parts of the
projects, such as the project origin, the community, communication and
structure of the project, the release process with its schedules and cycles, as
well as each project's development process with the general workflow,
development lead and feature inclusion process.

The following discussion took the results of the previous analysis and compared
it with similar researches or studies. High importance was given to a
comparison with other research findings and whether they are similar or if the
found results don't match with them.

As mentioned before the most important finding was that the project analysis
catalogue was adequate for a development process analysis of the projects.
Therefore the processes of the analyzed \ac{FOSS} project's aren't that
different after all. Of course, no project is equal to another one, but the
general processes exist in one form or another in all other projects.

Similarities can be found already in the project's inception phases, which all
started due to personal motivation. This generally concords with other
findings, too. An unexpected result was that the Bazaar model could not be
identified in any of the analyzed projects. Other researchers came to the same
conclusion, however the Bazaar model is closely related to \ac{FOSS} projects
and therefore one would anticipate it in most of \ac{FOSS} projects. On the
contrary, all projects had a hierarchical structure, even if they were modular
or had a flat structure. It has to be noted however, that only the core parts
of the projects were analyzed and therefore, a Bazaar like model could still be
identified around the projects. This was not analyzed in this thesis.

Despite the hierarchical structure of the projects, they are still very
welcoming to new developers and often one can quickly progress and become a
member in a leading group within the project. In this facet the Bazaar model is
reaffirmed.

Furthermore there was a noticeable move towards fixed release cycles and more
generally to an already pre-defined release schedule. This is very surprising
as it is a widely spread opinion that \ac{FOSS} projects do release when
\emph{it's ready}. This still can be seen in several projects, such as Debian
or Drupal, however both projects have a strong opinion for fixed release cycles
and it is only be a matter of time until they implement them.

Similar to commercial projects all analyzed projects had people or teams in
place to ensure the compliance with the release schedule. This finding is also
closely related to the previously noted hierarchical structure in the projects.

Another important finding is that the development processes of \ac{FOSS}
projects are similar between each other but cannot be resembled by traditional
or agile software engineering methods. The development processes were all very
agile and iterative. The outcome of this is obviously an evolutionary process
in which the product gets better with each cycle. There are only a few cases
known in which the project restarted from scratch. As such the processes and
development methods found were quite different to the previously described
software engineering methods. However several aspects of each proposed method
can be found in the engineering cycles.

Finally the finding by \textcite{Conway1968} can be corroborated and applied to
the findings. It states that the organization of a software system is similar
to the group which designed and implemented the system. According to the
findings in this thesis this not only holds true for the final product but also
for the development processes and workflows. The Python project stands as a
primary example of a very organized and structured group which also established
a very formal and textured development process.

By no means this analysis can be considered as exhaustive, as the findings only
hold for the chosen projects and their core parts. As such, it would be
interesting to apply the project analysis catalogue to an extensive set of
projects and analyze differences. A further analysis of such kind should of
course establish different project selection criteria, such as less or more
mature projects, but also smaller or larger projects than the analyzed ones.

Furthermore this analysis did not have a strong focus on the motivation of
developers, the community or interaction thereof. These categories could
provide material for an interesting study to see how they empower the different
structures or workflows.

Also, as noted before, projects seem to evolve parallel to their processes over
time. This raises a number of questions for future research such as if they
evolve with a similar process or if there are similar processes or workflows to
be found at similar times.

While the initial findings are promising, further research is necessary. This
also includes the assets and drawbacks of fixed release cycles and why projects
tend to adopt them.

Concluding this section, the work presented in this thesis provides a
consistent framework for analysis and discussion of \ac{FOSS} projects. Ideally
it is able to describe processes adequately with a high focus on reusability
and integrity. \acl{FOSS} projects are essential tools in the modern
daily grind and we should be eager to see what else we can learn from
\acl{FOSS} development processes and projects.

% }}}
