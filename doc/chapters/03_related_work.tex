\chapter{Related Work} % {{{
\label{chap:related work}

Although \ac{FOSS} is a quite new research subject, many different findings
exist in this domain. The most relevant to this analysis will be presented in
this chapter.

\section{Project Structure} % {{{

Certainly one of the most relevant and most known studies of \ac{FOSS}
development processes is \citetitle{Raymond1998} by \textcite{Raymond1998}. It
is an essay, based on the author's observation which contrasts two software
engineering models found in \ac{FOSS} development. The \emph{Cathedral} model
which is planned and developed by a small group of architects and the
\emph{Bazaar} model in which the project gets developed in a more chaotic way
by a number of developers with no real leader.

The research by \textcite{Capiluppi2007} draws on \citetitle{Raymond1998} and
the authors suggest that the Cathedral and Bazaar model are not mutually
exclusive, much more they offer the assertion that especially bigger \ac{FOSS}
projects go through both models starting with a Cathedral like development
model and then when becoming larger adopting the Bazaar model.

\textcite{Godfrey2000} on the other hand try to compare \ac{FOSS} development
processes with industrial and traditional management techniques. Additionally
they claim that especially big \ac{FOSS} projects have the ability to still grow
at least linearly.

\textcite{Kim2003} takes a more general approach and tries to describe the
\ac{FOSS} landscape as a whole, the demographics of developers and the
structure of \ac{FOSS} projects. The conclusion is that in most \ac{FOSS}
projects the development is actually led by a small group or single developers
which dissents a Bazaar like model. Interesting however is the open approach on
information sharing and easily accessible collaboration.

\textcite{Ogawa2007} visualized the communication in several established
\ac{FOSS} projects and came to the conclusion that a small group of people were
active in most of the discussion about development or future plans.

Also \textcite{Krishnamurthy2002} makes a strong case for a different
development model. In a case study including 100 mature projects the finding
was that most projects were indeed lead by small groups of people. Furthermore
a Bazaar like discussion was often not happening at all. However the project
age seemed to correlate with the number of developers and in addition the
number of project leaders decreased. Similar findings are proposed by
\textcite{Crowston2005a} who analyzed over 120 projects. They suggest however
that bigger projects tend to decentralize structures and projects do vary quite
drastically in terms of communication structure.

Based on a survey of over 2700 \ac{FOSS} developers \textcite{Ghosh2005}
assumes a classification of \ac{FOSS} projects which range from a hierarchical,
connected structure to a flat non-connected structure. A similar study which is
limited to the Debian project was conducted by \textcite{Sadowski2008}.

Finally however the research paper by \textcite{Conway1968} contains an insight
that the organization of a software system is similar to the group which
designed and implemented the system.

% }}}

\section{Motivation} % {{{

Many research papers exist about motivation of developers or contributers to
\ac{FOSS} projects. \textcite{Lakhani2002} for example examined why people do
provide free support to other developers or users. They claim that most people
do offer support because it returns direct learning benefits.
\textcite{Lerner2000} focus more on developers.

Also \textcite{Grazzini2009} questioned why developers would work for free and
came to the conclusion that several technological, social and economic factors
provide a complex interaction which is the reason for a developer's motivation.

\textcite{Lakhani2003} implemented a web based survey analyzing the answers of
over 600 developers and over 287 projects. They consider external motivational
factors as implausible and propose enjoyment-based intrinsic motivation as the
main motivation for professionals but also for volunteers. A similar study, but
limited to the Linux kernel was done by \textcite{Hertel2003}. The findings
were similar, however analyzed from a psychological point of view.

% }}}

\section{Software Engineering} % {{{

A descriptive analysis of \ac{FOSS} development is offered by
\textcite{Roets2007}. The authors claim that no single software development
process exists. Yet they derive a software development cycle based on different
established software engineering processes.

\textcite{Warsta2003} discuss whether agile methods and \ac{FOSS} development
methods are similar. They come to the conclusion that there is similarity,
however with their not inconsiderably distinctions. They suggest however that
both methods can learn from another. A similar finding is provided by
\textcite{Koch2004} who names the biggest difference the co-location in agile
methods, which of course is not available in most \ac{FOSS} projects.

\textcite{Spinellis2004} make a strong case that \ac{FOSS} affects traditional
software development since many \ac{FOSS} projects do actually share pieces of
code or use other projects for their development. A concrete analysis of
software engineering processes in the GNOME project is offered by
\textcite{German2003}.

Taking a look on the evolution of software engineering processes
\textcite{Scacchi2006} provides a study claiming that development processes
evolve together with the community of a project and that their development
processes have great influence on companies who were used to traditional
software engineering processes.

% }}}

\section{Case Studies} % {{{

In addition to the research studies, many case studies are relevant to this
analysis and focus on several \ac{FOSS} projects. \textcite{Mockus2002} analyze
Apache and Mozilla and compare them with several commercial projects.
\textcite{Dinh-Trong2004} provide a case study on the FreeBSD project with a
final comparison with the Apache project.

\textcite{crowston2004} analyzed several \ac{FOSS} projects and offered an
analysis of several project success measures. Another case study is provided by
\textcite{Magnusson2010} who considered only the PHP project. Similarly a case
study of the GNOME project is provided by \textcite{Koch2002}. Furthermore the
Plone project was analyzed in depth by \textcite{Aspeli2005}.

\textcite{Johnson2001} provides a quite exhaustive process model of different
\ac{FOSS} projects. He comes to the conclusion that most \ac{FOSS} projects
follow an adaptive life cycle. Almost all analyzed projects have established a
flexible management with it's key parts leadership, collaboration and
accountability.

A theoretical model of the structure of \ac{FOSS} projects is offered by
\textcite{Crowston2005}. The proposed model focuses on software development,
distributed work and the structure of distributed teams.

% }}}

% }}}
