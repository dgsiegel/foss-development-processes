\chapter{Development Process Analysis} % {{{
\label{cha:Development Process Analysis}

\section{Project Analyses Criteria} % {{{
\label{sec:Project Analyses Criteria}

% }}}

\cleardoublepage
\input{chapters/03_analysis_drupal}

\cleardoublepage
\input{chapters/03_analysis_plone}

\cleardoublepage
\input{chapters/03_analysis_python}

\cleardoublepage
\input{chapters/03_analysis_php}

\cleardoublepage
\input{chapters/03_analysis_gnome}

\cleardoublepage
\input{chapters/03_analysis_kde}

\cleardoublepage
\section{Definition and Origin of the Catalogue} % {{{
\label{sec:Definition and Origin of the Catalogue}

To characterize the development process of FOSS project, the following
catalogue was established. By covering all listed points one a project should
have been analysed thoroughly and lead to a reasonable breakdown of all
projects.

\begin{description}

  \item[Description of the Project] A general description about the project
    should be provided to inform the reader about the goals and current state
    of a project.

  \item[Project Category] In order to provide methods to compare projects, the
    project category can help to contrast projects of the same and other
    categories. Examples are programming languages or content management
    systems.

  \item[Scope of Analysis] As FOSS projects often are hard to analyze due to
    their size and unclear definition of modules, the scope will be limited to
    a well known subset of the whole project.

  \item[License] Some projects use their own license and some use the well
    known General Public License. It is interesting to find out whether this
    makes any difference in the development process of a module.

  \item[History] As only projects with a quite long history are analyzed it
    makes sense to glance at the project's history.

  \begin{description}

    \item[Founders] In some projects, the original founders are still present
      and have a influential voice, in others the original authors are no
      longer involved. To analyzed this further, the authors have to be
      introduced and their role will be analyzed in the community part.

    \item[Project Age] Projects always need time to evolve and to find their
      best development process. As this needs time, the project age can give
      some information in what development process state this project is and
      how it will evolve in the future.

  \end{description}

  \item[Community] The people behind the project are the driving force. Without
    them, a project would not exist. Therefore it is important to analyze the
    diverse community of a project.

  \begin{description}

    \item[Size of the Community] An important measure is the approximate size
      of a community. Several structural changes inside the project can depend
      on this size.

    \item[Communication inside the Project] The communication is a vital thing
      in an open project. It is interesting to compare the different methods of
      communication in several projects depending on their development
      structure and size.

    \item[Conferences and Meet-ups] Meetings of developers are an important
      element of the development process in FOSS projects. Also it shows, that
      there is enough interest available to provide the money needed for
      preparing such events.

    \item[Roles] The development process is often defined and lead by important
      roles in the project. Also, depending on what role a person has in the
      project, his influence varies. 

    \item[Role of the Founders] In some projects the founders are still
      actively involved and in some not. Depending on that, the founders often
      have a very important role in the project development process.

  \end{description}

  \item[Release Process] In the FOSS world an often heard statement is that a
    new version will be released when it is ready. In fact however quite all
    projects have a certain release schedule or process they follow.

  \begin{description}

    \item[Version Naming] Each project provides a specific version naming
      scheme to which they adopt and which characterizes major, minor or bug
      fixing releases. This information is vital to understand the project's
      release schedule.

    \item[Characterization of Major and Minor Releases] Most projects provide
      some kind of major releases with new features which is often incompatible
      with previous releases, minor releases which mostly are backwards
      compatible and provide new features and finally bug fixing only releases.
      This however differs from project to project.

    \item[Release Schedule] Even if there are no fixed release cycles, most
      projects have nevertheless a release schedule in action. The release
      schedule defines deadlines and important steps in the process of creating
      a new release.

    \item[Important Steps in the Schedule] These often define freezes or
      specific points in time when all members of the project are restricted to
      for example not provide new code in order to increase the stability of
      the new release.

  \end{description}

  \item[Development] The actual development is closely interweaved with the
    release process and defines how and when the development of a project takes
    place.

  \begin{description}

    \item[Development Lead] In fact, most FOSS projects are lead by groups of
      people who define new features and lead the development process.
      Additionally there are hardly any projects which have a completely
      unstructured development process and have no development leaders in
      place.

    \item[Development Workflow] The actual development process as often defined
      by the project leaders provides ways and methods to propose and develop
      new features for the upcoming release.

    \item[Feature Inclusion Process] This inclusion process is used by some
      projects which defines process how and when new features and come into
      the project.

  \end{description}

\end{description}

% }}}

% }}}
