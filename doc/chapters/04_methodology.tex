\chapter{Methodology} % {{{
\label{chap:methodology}

Based on the scientific research methods explained in \autoref{chap:theoretical
background}, this chapter will give an explanation of the used methodology,
research choices as well as the approach to the different projects.

\section{Project Selection} % {{{

As there is a huge number of \ac{FOSS} projects available, this study focuses
on well established projects with a quite long history compared to others. To
achieve a good selection of such projects several criteria were established to
ease the selection. Those criteria explain why only certain projects were
chosen to be analyzed. Furthermore they should give a reasonable selection
within this analysis can produce valid results.

\subsection{Category} % {{{

There is a vast number of \ac{FOSS} projects available. As of course not all
projects have the same goals in mind and provide similar software, it makes
sense to differentiate the projects by their goals and the type of the
resulting software. Examples are programming languages, desktop interfaces and
other. Such a breakdown makes it possible to directly compare projects with
similar goals as well as with other, not related projects.

% }}}

\subsection{Popularity} % {{{

To not choose small or not relatively unknown projects, the popularity and
usage of a project was taken into consideration. While it is very difficult to
represent this factor with absolute numbers, the number of active developers,
approximations of installations and citations in websites or magazines gives a
good quantitative representation.

% }}}

\subsection{Project Age} % {{{

It seems that especially \ac{FOSS} projects take some time to identify their
development workflow and structure their project accordingly. Furthermore quite
new projects often do not have such a structured development workflow and it is
seldom known if they still will be developed in the near future. As such a
minimum project age was set to ten years. To exclude older projects, which are
only maintained and do not follow a typical \ac{FOSS} workflow the maximum
project age was set to 25 years.

% }}}

\subsection{Activity} % {{{

The activity of a project is a very important measure to include only actively
developed projects which still fall into the above mentioned time frame. The
activity of a project however is difficult to measure, but one can look at
certain numbers published by most of the projects which allow a good insight of
how active the project is without rating it quantitatively.

\paragraph{Releases} % {{{

Regular or a high number of releases are a sign of a high activity of \ac{FOSS}
projects. As such this represents an important criteria for the selection.

% }}}

\paragraph{Downloads} % {{{

Some projects publish the number of accesses to their code repositories or
number of downloads of their releases. A big number of course shows a high
popularity and also a high activity within the project.

% }}}

\paragraph{Number of Commits} % {{{

This is a correlation between the number of commits to a project's repository
and it's activity. As such a large number of commits shows also a high
activity.

% }}}

% }}}

\subsection{Community} % {{{

The people behind a project are the driving force and directly lead and develop
a project. As such it makes sense to provide some criteria to choose projects
with a considerable number of developers.

\paragraph{Communication} % {{{

In order to be able to analyze development workflows in a project, the
communication methods have to be openly accessible. Furthermore a project
should have a good level of communication over diverse methods such as mailing
lists, forums or chat systems.

% }}}

\paragraph{Developers} % {{{

The number of developers of a project is a direct indicator of the popularity
of a project, it's size and activity. As such a large number of developers is
wanted for this analysis, however it is not always possible to get an accurate
number.

% }}}

\paragraph{Conferences} % {{{

Meetings such as conferences, hackfests or other are an indicator of an active
and vital community. They represent an important meeting point in most projects
and as such are a good criteria for the activity of projects.

% }}}

\paragraph{Foundations} % {{{

Bigger \ac{FOSS} projects mostly have a foundation as backer, which makes sure
that the project stays independent from companies or other. They also appear
after a few years of development and a larger size of the project.

% }}}

\paragraph{Ongoing Projects} % {{{

A lot of bigger \ac{FOSS} projects hold or join projects like Google's Summer
of Code program in which students are invited to join the project for a few
months. Such programs almost always lead to a larger community which is able to
host such events.

% }}}

% }}}

% }}}

\section{Final Selection} % {{{

With the above explained criteria catalogue, ten \ac{FOSS} projects were chosen
and analyzed. This leads to the following list of projects.

\begin{table}[h!]
  \centering
  \begin{tabularx}{\textwidth}{llX}
    \toprule
    \tableheadline{Project} & \tableheadline{Age} & \tableheadline{Category} \\
    \midrule
    Debian        & 1993 & Operating System \\
    Drupal        & 2001 & Content Management System \\
    Fedora        & 2002 & Operating System \\
    GNOME         & 1997 & Desktop Environment \\
    KDE           & 1996 & Desktop Environment \\
    MySQL/MariaDB & 1997 & Database Management System \\
    PHP           & 1994 & Interpreted Programming Language \\
    Plone         & 1999 & Content Management System \\
    PostgreSQL    & 1986 & Database Management System \\
    Python        & 1989 & Interpreted Programming Language \\
    \bottomrule
  \end{tabularx}
  \caption[List of analyzed \acl{FOSS} projects]{List of analyzed \ac{FOSS} projects.}
\end{table}

A deliberate choice was made to only analyze the core parts of the projects,
even if they are built with a modular approach or consist of several other
parts. This choice was made to allow an easier comparison between the projects
without any clutter they might bring in.

% }}}

\section{Visualization of Project Data} % {{{

For this analysis not only the publicly available project information was
analyzed, but also the code repositories of the projects. The reason for this
is underly the made conclusions and to check the analysis for validity. For
\ac{FOSS} projects all code repositories are publicly available and often range
back to the beginning of the project. In some cases however data from the
beginning is not available, be it because of the lack of versioning control
systems at that time or migrations of version control systems. In other cases
the code was split into several different code repositories, which made it
harder to combine the given data. With those two points in mind, all software
repositories were downloaded and prepared for automatic analysis. This produced
several graphs which will be explained below.

\subsection{Commits by Author} % {{{

\begin{figure}[h!t]
  \centering
  \includegraphics[width=0.8\textwidth]{python/commits_by_author}
  \caption{The graph displaying the six most active developers with their
  monthly commit number with data from the Python project.}
\end{figure}

The above graph shows the six most active developers of a project with their
over time commits per month. It allows to quickly see the involvement of
certain people in the project and their activity on the development side of the
project. In this case for example the Python project leader Guido van Rossum
diminished his activity from 2004 on in comparison to other developers. This
could mean two things, first a person could have left a project or, which is
the case in the Python project, the person is busy in other parts of the
project.

For the generation of this graph all related software repositories were
analyzed in relation to each developer with the most commits. The first six
were then chosen and their monthly commit count was analyzed from the beginning
of the projects inception (if available) till the end of the analysis. The
monthly number was then plotted and a curve was interpolated between these
points. To smooth the curve and to allow an easier understanding of the graph
each second month was left out and interpolated over the other points.

% }}}

\subsection{Commits by Year} % {{{

\begin{figure}[h!t]
  \centering
  \includegraphics[width=0.8\textwidth]{kde/commits_by_year}
  \caption{A graph displaying the number of commits per year with data from the
  KDE project.}
\end{figure}

The above graph shows the number of total commits per year. Most projects have
a steadily increasing number over the years. In some cases like the above there
is a quite large leap. In the above case this is simply explained with the
development of a new major KDE release.

The graph was generated counting all commits per year from inception till the
end of the analysis. It is interesting to note that this graph often matches up
with other graphs in the same project.

% }}}

\subsection{Time Based View} % {{{

\begin{figure}[h!t]
  \centering
  \includegraphics[width=\textwidth]{gnome/punchcard}
  \caption{A punchcard graph showing when commits usually occur with data from
  the GNOME project.}
\end{figure}

This graph shows all commits from inception till the end of this analysis in
form of a circle with the related day and time combination. The bigger a circle
is, the more commits, related to the other day/time combinations, occurred. In
the below example the most busy day was Monday between \unit[4]{pm} and
\unit[12]{pm}. This can be easily explained with the fact that the GNOME
project usually does new releases on Monday. Furthermore such a graph shows,
the developers usual working hours. This example also shows that many
developers tend to have a main job in a company, as usual office hours match up
with the day and time combinations. This is enforced by a quite low number of
commits on Saturday and Sunday in comparison to workdays.

The graph was generated counting the commits depending on their day and time.
Of course local time was converted to \ac{UTC} to guarantee comparable results.
The circles were then drawn on the graph using different sizes and transparency
depending on the number of commits on a single day and time combination. The
size and transparency is a relative value ranging from the day and time
combination with the least number of commits to the day and time combination
with most commits. All other circles were then drawn using a value between this
range.

% }}}

\subsection{Commits by Month} % {{{

\begin{figure}[h!t]
  \centering
  \includegraphics[width=0.8\textwidth]{drupal/commits_by_month}
  \caption{A graph showing the number of commits per month with data from the
  Drupal project.}
\end{figure}

The above graph shows the commits per month correlation from inception till the
end of this analysis. It is quite similar to the yearly overview mentioned
before, however gives a better insight into the monthly development. An
increase of commits always shows a certain degree of interest of people for the
project. It can of course mean that more people are willing to contribute to a
certain project, but also that the existing developers are producing more code.
This is especially interesting when comparing it with the following graph. In
the above example there is a peak between 2008 and 2009 which can be explained
with the upcoming Drupal 7 release development at that time .

The graph was generated counting the commits from inception till the end of
this analysis and plotting that number for each month. The blue line then was
interpolated between those points. The red line on the other hand stands for
the average over three months and is useful to identify the project's
direction. As one can see it starts with the same value as the blue curve has
and applies the average over three months starting with the first available
month.

% }}}

\subsection{Authors by Month} % {{{

\begin{figure}[h!t]
  \centering
  \includegraphics[width=0.8\textwidth]{php/authors_by_month}
  \caption{A graph showing the number of distinct authors per month with data
  from the PHP project.}
\end{figure}

The above graph shows the number of distinct authors per month from inception
till the end of analysis. It gives a good impression on how many people were
responsible for the amount of work done in a certain time period, which can be
seen in the previous graphs. The above example shows a decrease of authors
meaning either that developers left the project or that developers reduced
their time working on the project.

The graph was generated counting all distinct authors per month from inception
till the end of analysis. As this is always an integer, the interpolation is
done over a so called constant plot. The red line on the other hand is an
average value over three months which is interpolated between those points. In
case of authors with different email addresses or slightly different names in
the single commits, they were combined if it was the same person.

% }}}

% }}}

% }}}
