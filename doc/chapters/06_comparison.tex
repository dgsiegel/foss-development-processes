\chapter{Comparison of Development Processes} % {{{
\label{chap:comparison}

As the analyzed projects were described in the previous chapter, this chapter
will compare the development processes and project details with each other.

\section{Project Origin} % {{{

All analyzed projects share a quite different history. Some were developed with
an university background like PostgreSQL or Python while other projects were
born in the \ac{FOSS} context. The GNOME or Debian project would be an example
for that. Projects like Plone or Fedora were an improvement and additional
layer to an existing project.

It is interesting to note, that many project founders tried to solve a personal
problem with the project they were about to start. KDE, Drupal, Fedora and PHP
might be notable here. However also commercial or philosophical background can
initiate a project, such as MySQL with MySQL AB or GNOME show.

\begin{table}[htb]
  \centering
  \begin{tabularx}{\textwidth}{lX}
    \toprule
    \tableheadline{Project} & \tableheadline{License} \\
    \midrule
    Debian        & Various, mainly \ac{GPL} \\
    Drupal        & \ac{GPL} \\
    Fedora        & Various, mainly \ac{GPL} \\
    GNOME         & \ac{GPL}, \ac{LGPL} \\
    KDE           & \ac{GPL}, \ac{LGPL} \\
    MySQL/MariaDB & \ac{GPL} \\
    PHP           & PHP License \\
    Plone         & \ac{GPL} \\
    PostgreSQL    & PostgreSQL license \\
    Python        & Python Software Foundation License \\
    \bottomrule
  \end{tabularx}
  \caption{Used licenses in analyzed projects}
\end{table}

Even if the number of developers increased over time, all projects were started
by a small group of people which often consisted of at most two persons. Often,
a single person was more involved in the project creation, even if other people
were involved too. Michael Widenius of MySQL/MariaDB would be a fitting match
to notice. After inception however the number of involved people grew rapidly.
This holds for all analyzed projects and is especially the case when the
project was finally released as \ac{FOSS}. Examples like PostgreSQL or Python
show the rapid increase of involved people, be it just developers or other
involved people, when finally released as \ac{FOSS}. The growth graph however
is quite similar in all analyzed projects, if one considers time and size
differences between the projects. Also it is worth noticing that almost every
analyzed project has one or more rapid developer growth periods. Those periods
can be matched with a certain major release of the projects. This becomes clear
when one examines the release and development of Python 2.0, GNOME 3.0, KDE 4.0
and Plone 4.0. All of those versions are major releases with new features and
approaches to their project goals. Apparently the development of such big
releases attracted a lot of new developers. Furthermore the size stayed quite
the same after the release.

\begin{table}
  \centering
  \begin{tabularx}{\textwidth}{llX}
    \toprule
    \tableheadline{Project} & \tableheadline{Age} & \tableheadline{Founder} \\
    \midrule
    Debian        & 1993 & Ian Murdock \\
    Drupal        & 2001 & Dries Buytaert \\
    Fedora        & 2002 & Warren Togami \\
    GNOME         & 1997 & Miguel de Icaza \& Federico Mena Quintero \\
    KDE           & 1996 & Matthias Ettrich \\
    MySQL/MariaDB & 1997 & Michael Widenius \& David Axmark \\
    PHP           & 1994 & Rasmus Lerdorf \\
    Plone         & 1999 & Alexander Limi \& Alan Runyan \\
    PostgreSQL    & 1986 & Lawrence A. Rowe \& Michael R. Stonebraker \\
    Python        & 1989 & Guido van Rossum \\
    \bottomrule
  \end{tabularx}
  \caption{Project founders and origins}
\end{table}

% }}}

\section{Community} % {{{

The previous project analysis has shown, that the communities of each project
seem to be quite different. This will be compared in the following sections.

\subsection{Size of the Community} % {{{

As all of the analyzed projects share a quite long history, a lot of people
were active in the projects. This of course biases the comparison between
communities, as the actual number of developers is quite always lower than the
total sum of developers in the project history. Also taking in mind that only
the central parts of the projects were analyzed, the resulting number can vary
a lot. For example the GNOME project claims that over 3500 people contributed
to the project. The fact that only GNOME core was analyzed and not anymore
active developers were left out reduces the finding to currently about 350
active developers of GNOME core. The same probably holds true for the Drupal
project with about 1200 enlisted developers and for the KDE project with about
1800 active contributors.

Also it is important to distinguish carefully between core developers and other
contributors. As in some projects only core developers have access to the code
repository, an analysis of code authors is not always a satisfying approach.
The Drupal project for example only has about six to ten distinct authors to
Drupal core. Only recently they adopted to give credit to the original author
instead of referring to it in the commit message. Another interesting insight
is the Plone project which claims to have about 300 active core developers, the
code repository however only shows a number of the dimension from 50 to 70
active developers at the same time. The PostgreSQL project on the other hand
gives an exact number of active and retired core developers, only leaving out
the number of minor contributors.

It is not necessarily the case that the actual number of developers is smaller,
projects like Fedora or Debian with many different aspects of work and smaller
tasks like package creation can actually have a number of active developers
which quite matches the claimed number by the project. However one has to keep
in mind that the comparison of a core piece of a project with a full operating
system might be unfair and not very significant. That said the size comparison
has to be made carefully with the above considerations in mind.

% }}}

\subsection{Communication} % {{{

On a worldwide distributed project the communication inside a project is one of
the most vital parts. Every project has therefore a wide array of communication
methods in place. Quite every project uses mailing lists as it's favourite
communication method. A plausible explanation for this would probably be the
fact that it is easily scalable, open and archived.

Although every project has lots of different lists for quite every aspect in
the project, there is always one mailing list which is the most important one.
This is true for GNOME with the \emph{desktop-devel}, PHP with the
\emph{internals}, Plone with the \emph{plone-developers}, Python with the
\emph{python-dev}, PostgreSQL with the \emph{pgsql-hackers}, Fedora with the
\emph{devel} and Debian with the \emph{debian-devel} mailing list. In some
projects this single mailing list is split up into two, one more general for
the project and the second for the core project. This is the case for KDE with
the \emph{kde-devel} and \emph{kde-core-devel} mailing lists and MariaDB with
the \emph{maria-developers} and \emph{maria-captains} mailing lists. An
exception of this might be the Drupal project which uses mailing lists however
the most important communication channel is a quite mailing list akin forum on
their project website.

Additionally in every project there are lots of other direct and indirect
communications such as \ac{IRC} channels, blogs, newsletters and other in
place.

% }}}

\subsection{Conferences} % {{{

As most of the developers are living in different cities, countries or
continents, the communication is almost always done over the internet.
Therefore conferences and meet-ups provide the possibility to meet in person,
plan and design future goals more easily and to invigorate the community. Each
analyzed project has one or more annual conferences in place where developers
and users can meet. It is interesting to note that it always took some years to
get to an steady frequency. Also it seems that most projects had some years of
development behind them before the first conference was held. Along with the
growth of a project the number of attendees at a conference grew as well.
Another interesting fact might be that some projects include the conference in
their release and development schedule, for example the GNOME project in which
the conference should provide an extra development boost for the upcoming
release in autumn.

Due to the size of the project and the worldwide distribution of developers,
each project has also a lot of local meet-ups, hackfests, sprints, smaller
conferences or other workshops in place. They however focus more on specific
parts or goals of the project.

% }}}

\subsection{Roles} % {{{

The analysis of the projects has shown, that all projects use a fairly
hierarchic development structure. Starting from the bottom there is always a
role for simple developers or contributors. Going the hierarchy upward, one
gets more rights but also more responsibilities. For example one could manage
certain parts of the project or provide guidance to new developers. Above the
maintainers of a certain subsystem is rather always a small group managing the
project's development. Then, at the top one can find a project leader or a
group engaging the project leader role. It is also interesting that some
projects split the same function into two roles while others only use one.
Keeping this fact and the different project scopes in mind, the used structures
are middlingly similar, especially if the projects are working in the same
field.

\begin{figure}[htbp]
  \centering
  \subfloat[Drupal]
    {\includegraphics[width=.35\textwidth]{drupal/roles}} \qquad
  \subfloat[Plone]
    {\includegraphics[width=.35\textwidth]{plone/roles}} \\

  \subfloat[Python]
    {\includegraphics[width=.35\textwidth]{python/roles}} \qquad
  \subfloat[PHP]
    {\includegraphics[width=.35\textwidth]{php/roles}} \\

  \subfloat[GNOME]
    {\includegraphics[width=.35\textwidth]{gnome/roles}} \qquad
  \subfloat[KDE]
    {\includegraphics[width=.35\textwidth]{kde/roles}} \\

  \subfloat[PostgreSQL]
    {\includegraphics[width=.35\textwidth]{postgresql/roles}} \qquad
  \subfloat[MariaDB]
    {\includegraphics[width=.35\textwidth]{mariadb/roles}} \\

  \subfloat[Fedora]
    {\includegraphics[width=.35\textwidth]{fedora/roles}} \qquad
  \subfloat[Debian]
    {\includegraphics[width=.35\textwidth]{debian/roles}} \\
  \caption{Comparison of the development related roles in the projects.}
\end{figure}

Of course this structure varies over time and size of the project. The Debian
project for example which has a very long history in comparison to the others
has a quite filed segmentation with clear duties and responsibilities for each
role. A younger project like MariaDB has a simpler model, which nonetheless
has the described groups on top.

A release team seems to be very important as it exists in all projects, either
as a dedicated team or is represented by the project leader. Furthermore the
concept of having people who are responsible for certain subsystems seems to be
established widely as all bigger projects use it.

It is also interesting to note, that in projects where the project founder is
still active, the structure is formed more into a hierarchy with the founder on
top. This is true for Drupal, PHP and Python. Debian is a bit of a special case
here because the original project leader passed on his duty preserving the
project leader hierarchy. Sometimes the leader gathers other people around him
trying to be not solely responsible for the project, such one can see in the
MariaDB project. In projects, the leader is no longer involved or at least does
not hold any power, project leader teams were established. This holds for KDE,
Fedora and GNOME.

Even if projects claim to be friendly to newcomers and an open community, the
structure analyzed here shows, that one always has to start at the bottom of
the hierarchy, earning trust and providing enhancements to the projects before
being able to become a leader or holding a carrying business. 

% }}}

\subsection{Project Founders} % {{{

The project founders played a vital role in the inception of the project. It
was already explained that even if there are more project founders, often one
single person plays the most important role. It was also pointed out that some
project founders are still active and lead the project, while others have left
their project.

Therefore the project founders of the analyzed projects can be classified into
three categories. First, project founders which are still active in the project
and have a leadership position. This is the case for the Drupal project, as
well as for Python and PHP. In all of those projects, the development structure
might look different without the leader. They define goals, releases, new
features and is often the deciding person. The opposite are projects where the
original founder no longer is present. Fedora, Debian and PostgreSQL are good
examples here. All named projects established a leadership group which defines
the future of the project. Debian still has a project leader, who however gets
voted by the community. At last there are projects, where the original leader
is still around and sometimes gives input but has no power. In the KDE and
GNOME projects the founders are working on related projects or topics and
sometimes give their opinion, but they can't decide anything. In the Plone and
MariaDB project the founders are members of the leadership team but have no
special role in there. At most they can give their opinion and community
members might give their notion a higher value.

% }}}

% }}}

\section{Release Process} % {{{

The release process seems to be the most diverse item in this analysis, as most
projects have gone through different processes and workflows during their time
of existence. Nevertheless some of the different stages of the projects seem to
be quite equivalent to each other.

\subsection{Version Scheme} % {{{

All projects, except Drupal and Fedora, use a three digit version naming
scheme. It consists of a major, minor and micro number, which characterizes a
release. The Drupal project uses a two digit versioning scheme, which defines
major and minor releases. It is quite important to note, that the minor
releases of a major release only include bug fixes and therefore are more
following the scheme of a major and micro version number. The Fedora Project
just uses a single number to represent major releases. Due the structure and
automatic update process, bug fixes can be included automatically in a running
system. For comparison the Debian project also pushes bug fixes to the users,
however it additionally releases those changes as further bug fix releases.

It is true for all projects, that an increment in the major number defines an
often backwards incompatible release with new features. Such a major release is
often a ground breaking change and often defines a new era for the project.
They are also extremely scarce. This is also reflected by the comparatively low
major number the projects have, for example GNOME 3, KDE 4, Plone 4, Drupal 7
and so on. Fedora is a bit of an exception here, as every new release
increments the major number by one.

The minor number almost always defines a feature release which is backwards
compatible to the previous major release. There are some exceptions where this
is not the case, for example in the PHP 5.x release cycle which. However those
changes can be seen as minor when comparing them to the incompatibilities of
two major releases. Also, if such changes occur, most projects provide an
upgrade path. Except Drupal and Fedora, all projects use such a number.

In some projects, an increment of the minor version number defines a major
release. In other words, regardless of whether the major or minor number
changes the release is always a major release. This holds for GNOME, PHP and
Debian.

The GNOME project always increments it's minor number by two to define a stable
release cycle. An odd number as the minor number always defines an in
development release.

Micro releases are a number of releases for the above mentioned feature
releases. They are only allowed to include bug and security fixes, no new
features and no incompatibilities. Except Fedora, all project use such a
number.

In a nutshell it can be said that all projects use a similar versioning scheme
with some minor exceptions, which do not have an all to big impact.

% }}}

\subsection{Release Schedule} % {{{

The release schedule leads to new major releases and often includes minor and
bugfix releases. They are therefore a vital point for each project's future. As
such most projects have established a rather detailed release schedule which
they try to follow accurately. Nevertheless there are some differences
comparing each project's schedule which will be outlined in the following.

The most distinguished difference is of course the decision of a fixed release
cycle. A fixed release cycle iterates over a given time frame and provides
expectable release dates. Projects like Plone, GNOME and KDE projects are using
a fixed release cycle with a duration of 6 months. It means, that each year
there will be two major releases including a number of minor or bugfix
releases. Also Fedora uses a similar cycle, as it is more or less bound to the
GNOME release cycle. The PostgreSQL and PHP projects use a fixed release cycle
with a duration of one year.

Although projects like Drupal and Python do not use a fixed release cycle for
major releases, they do however use a fixed release cycle for minor and bugfix
releases. The Python project publishes minor releases roughly every 18 months
and bugfix releases roughly every six months. The Drupal project publishes
bugfix releases monthly.

Only MySQL/MariaDB and Debian do not have a fixed release cycle, however at
least the Debian project has some effort going on to provide a fixed release
schedule.

All projects however have some sort of freezes of part of the project, such as
code, features, translatable strings, documentation and more. Freezes should
help to confine areas where new bugs or drawbacks could appear. Of course they
are restricted to new development and developers can still apply fixes for
bugs. However they will be analyzed deeply before they can be applied. As such
freezes are a vital part of all analyzed project schedules. Also, they often
keep in effect for the released major branch. One of the most obvious
consequences is the fact that bugfix releases only include fixes and no new
features or user interface changes.

\begin{figure}[htbp]
  \centering
  \subfloat[Drupal]
    {\includegraphics[width=.46\textwidth]{drupal/cycle}} \qquad
  \subfloat[Plone]
    {\includegraphics[width=.46\textwidth]{plone/cycle}} \\

  \subfloat[Python]
    {\includegraphics[width=.46\textwidth]{python/cycle}} \qquad
  \subfloat[PHP]
    {\includegraphics[width=.46\textwidth]{php/cycle}} \\

  \subfloat[GNOME]
    {\includegraphics[width=.46\textwidth]{gnome/cycle}} \qquad
  \subfloat[KDE]
    {\includegraphics[width=.46\textwidth]{kde/cycle}} \\

  \subfloat[PostgreSQL]
    {\includegraphics[width=.46\textwidth]{postgresql/cycle}} \qquad
  \subfloat[MariaDB]
    {\includegraphics[width=.46\textwidth]{mariadb/cycle}} \\

  \subfloat[Fedora]
    {\includegraphics[width=.46\textwidth]{fedora/cycle}} \qquad
  \subfloat[Debian]
    {\includegraphics[width=.46\textwidth]{debian/cycle}} \\
  \caption{Schematic representation of major and minor development cycles of
  the analyzed projects.}
\end{figure}

The most interesting fact seems to be when the freezes occur and with which
restrictions. The GNOME project as well as the Debian project have established
a freeze shortly before their final release including code, feature, user
interfaces, \ac{API} and string freezes. The freeze period of the GNOME project
however has a defined start and end date, as the release schedule is a fixed
one. The KDE project on the other hand has the different freezes much more
spread out through the release cycle. It also makes a difference between soft
and hard freezes. The freezes are set in order and always have a soft freeze
before a hard freeze. This applies for features, strings and \ac{API} freezes.
The Plone project is set somewhat in the middle having a feature freeze two
months before the final release and then applying more and more restrictions
during the releases of alpha, beta and release candidate versions. The Fedora
is quite similar to that approach, however has it's feature freeze roughly
three months before the final release. Following an alpha and beta release more
restrictions apply.

The Python project only allows new features until the first beta release which
occurs about two months before the final release. After that point only bug
fixes can be applied.

The PHP project has changed the layout of it's schedule a lot during the last
years on the way to a fixed yearly release schedule. Currently the feature
freeze as in act about eight months before the final release, confining other
freezes more and more until the first alpha release. Further beta releases
until the final releases did only get bug fixes from there on.

The not fixed release schedule of MySQL/MariaDB forces the project to establish
criteria when specific releases can happen. As such a criteria for a feature
freeze is the publishing of alpha releases. After that no new features are
allowed and only bug fixes get applied. This continues until the final release
is considered to be bug free.

Like MySQL/MariaDB the Drupal project does not have a fixed release cycle and
therefore arranges the development and freezes into different phases. Each
phase comes with it's freezes, such as code, \ac{API}, feature, user interface
and string freeze in this order. After all freezes have been declared a number
of alpha, beta, release candidate and last the final release will be published.

The PostgreSQL has replaced a single feature freeze with a series of commit
fest which were explained earlier. In fact they are the only analyzed project
which does not have a single freeze but a series of development milestones
which limit new features and development time by time.

It is also interesting to note, that every project uses some kind of alpha,
beta and release candidate versions before publishing the final release. The
naming is of course different. For example the GNOME project does not provide a
special naming, except for the odd minor number, while other projects, such as
Drupal or Debian name their release accordingly.

% }}}

% }}}

\section{Development} % {{{

While the release cycle defines the process around new releases, the
development process of a project defines the actual shape of a project. As such
it is vital to understand how projects lead, divide and structure their work.

\subsection{Development Lead} % {{{

The development of a project is always lead by a single person or a group of
people. In the analyzed projects there was no clue that the development just
happened without a form of leadership. The opposite was the case, each project
showed a hierarchical structure when it came to decisions about the development
of a project.

Often the development is lead by the founder, if he is still in the project.
In the Python, Drupal and PHP projects the original founder has a great
influence when it comes to decide on changes or new features. Projects like
Fedora, MariaDB, Plone and PostgreSQL have established leadership groups who
define the direction of the project as well as deciding over new features.
While the Fedora and Plone project have an arbitrary or elected group they do
not necessarily provide enhancements themselves. The MariaDB and PostgreSQL
projects on the other hand are mostly enhanced and lead by the same group.

While the KDE and GNOME projects have established core or release teams they do
not necessarily have power on areas other than their duties. Also due to their
modular structure, the concrete development is mostly lead by the maintainers
of single areas. A minor exception might be GNOME's design team which is
involved in the shaping of user interfaces. As however cannot dictate other
GNOME members what to do, the above assumption still holds.

The Debian project is structured very hierarchically and as such the overall
direction is set by the elected project leader. The technical committee then
looks after the concrete implementation while the single team leaders are in
place for certain areas of the project.

% }}}

\subsection{Development Workflow} % {{{

The development workflow of a project defines how changes and enhancements are
brought into the project. Some of the analyzed projects have established a
feature inclusion process which will be highlighted in the next section.

In all analyzed projects it is possible to commit small patches and
enhancements directly to the project's repository or let it commit by a core
contributor if one has not enough rights. Such changes are not tied to an
existent feature inclusion process. As such this often only comprises bug fixes
and other changes of that size. This is often assisted by a bug tracking system
each of the analyzed project has.

It seems that all changes committed to the repositories are reviewed by other
members of the project. In case of not gaining acceptance the patch is often
discussed in the project's mailing lists or other suitable means of
communication. It is also possible that patches are sent to the mailing list or
to a bug tracking system and discussed there before being applied. Quite all
projects have some sort of team leaders or maintainers who can decide about a
certain patch for a certain subsystem of the project.

% }}}

\subsection{Feature Inclusion Process} % {{{

The feature inclusion process defines, if established, how and when features or
big changes come into a project. In the analyzed projects a quite diverse
passel of processes has been found. It ranges from a very structured and well
documented process like the Python project has to a more dynamic approach like
the KDE project uses.

The most structured and well laid-out process is Python's \ac{PEP} process. It
defines a well documented approach of new features and big changes to the
project from it's inception until either acceptance or withdrawal. This process
seems to work well for the Python community as well for development, community
and leadership processes.

Interestingly other project's started to adopt this process with minor changes.
The \ac{PLIP} process which is used by the Plone project for example only uses
it for development enhancements and not for any community issue. As such the
process is a bit simpler and more focused to solely development than the
\ac{PEP} process. 

Also the PHP project adopted a version of the \ac{PEP} process which is called
\ac{RFC} process. As it also only focuses on development it is more similar to
the \ac{PLIP} process. A minor difference might be the separate voting step
over each proposal, which is not existent in the other two.

A bit more dynamic approach is the Drupal initiative process where each Drupal
developer can propose an initiative with a certain goal. At a certain point in
time the Drupal project founder and leader Dries Buytaert chooses the
initiatives the project should follow for the next release.

Similar to this, but a bit more structured and not backed by a single person is
the feature based development process of GNOME and Fedora. In both projects,
features can be proposed which are then chosen by the GNOME release team and
the \ac{FESCo}. In both projects the proposal follow a certain format and are a
bit more structured when compared to Drupal initiatives. Also, in both projects
the named groups have the possibility to drop or postpone a certain feature if
it is not ready for the upcoming release.

\begin{figure}[htbp]
  \centering
  \subfloat[Drupal]
    {\includegraphics[width=.48\textwidth]{drupal/feature}} \quad
  \subfloat[Plone]
    {\includegraphics[width=.48\textwidth]{plone/feature}} \\

  \vspace{0.5cm}

  \subfloat[Python]
    {\includegraphics[width=.48\textwidth]{python/feature}} \quad
  \subfloat[PHP]
    {\includegraphics[width=.48\textwidth]{php/feature}} \\

  \vspace{0.5cm}

  \subfloat[GNOME]
    {\includegraphics[width=.48\textwidth]{gnome/feature}} \quad
  \subfloat[KDE]
    {\includegraphics[width=.48\textwidth]{kde/feature}} \\

  \vspace{0.5cm}

  \subfloat[PostgreSQL]
    {\includegraphics[width=.48\textwidth]{postgresql/feature}} \quad
  \subfloat[MariaDB]
    {\includegraphics[width=.48\textwidth]{mariadb/feature}} \\

  \vspace{0.5cm}

  \subfloat[Fedora]
    {\includegraphics[width=.48\textwidth]{fedora/feature}} \quad
  \subfloat[Debian]
    {\includegraphics[width=.48\textwidth]{debian/feature}} \\

  \vspace{0.5cm}

  \caption{Schematic and simplified representation of feature inclusion
  processes of the analyzed projects.}
\end{figure}

The Debian project uses goals which can be proposed by community members and
are chosen by the release team. Antithetic to the previous projects the goal is
however just a single wording with no structure at all. No concrete steps,
involved parties, code or other are trailed to the goal.

The KDE and MariaDB projects use an even more dynamic approach where members
can propose features, but they have to picked up and implemented by developers.
As such there is no real instance who could prohibit the development of a
certain feature.

The PostgreSQL project is the only project which has removed such a specific
inclusion process altogether and replaced it with a series of commit fests. A
feature can be brought into a fest and evolves over time until it can be
committed. It can be said, that such a series of feature reviews is a series of
feature inclusion processes where however the goal is always to get a certain
enhancement into the project instead of discarding it directly.

% }}}

% }}}

% }}}
