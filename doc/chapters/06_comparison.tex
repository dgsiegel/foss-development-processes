\chapter{Comparison of Development Processes} % {{{
\label{chap:comparison}

As the analyzed projects were described in the previous chapter, this chapter
will compare the development processes and project details with each other.

\section{Project Origin} % {{{

All analyzed projects share a quite different history. Some were developed with
an university background like PostgreSQL or Python while other projects were
born in the \ac{FOSS} context. The GNOME or Debian projects are an example for
that. Projects like Plone or Fedora were an improvement and additional layer to
an existing project.

It is interesting to note, that many project founders tried to solve a personal
problem with the project they were about to start. KDE, Drupal, Fedora and PHP
are notable here. However also commercial or philosophical background can
initiate a project, such as MySQL with MySQL AB or GNOME show.

\begin{table}[htb]
  \centering
  \begin{tabularx}{\textwidth}{lX}
    \toprule
    \tableheadline{Project} & \tableheadline{License} \\
    \midrule
    Debian        & Various, mainly \ac{GPL} \\
    Drupal        & \ac{GPL} \\
    Fedora        & Various, mainly \ac{GPL} \\
    GNOME         & \ac{GPL}, \ac{LGPL} \\
    KDE           & \ac{GPL}, \ac{LGPL} \\
    MySQL/MariaDB & \ac{GPL} \\
    PHP           & PHP License \\
    Plone         & \ac{GPL} \\
    PostgreSQL    & PostgreSQL license \\
    Python        & Python Software Foundation License \\
    \bottomrule
  \end{tabularx}
  \caption[Used Licenses in the Analyzed Projects]{Used licenses in the analyzed projects.}
\end{table}

Even if the number of developers increased over time, all projects were started
by a small group of people which often consisted of at most two people. Often,
a single person was more involved in the project creation, even if other people
were involved too. Michael Widenius of MySQL/MariaDB fits this case. After
inception however the number of people involved grows rapidly. This statement
holds for all analyzed projects and is especially the case when the project was
finally released as \ac{FOSS}. Examples like PostgreSQL or Python show the
rapid increase of people involved, be it just developers or other , when
finally released as \ac{FOSS}. The growth graph however is quite similar in all
analyzed projects, if one considers time and size differences between the
projects. Also it is worth noticing that almost every analyzed project has one
or more rapid developer growth periods. Those periods can be matched with a
certain major release of the projects. This becomes clear when one examines the
release and development of Python~2.0, GNOME~3.0, KDE~4.0 and Plone~4.0. All of
those versions are major releases with new features and approaches to their
project goals. It seems that the development of such big releases attracts a
lot of new developers. Furthermore the increased size stayed almost the same
after the release.

\begin{table}
  \centering
  \begin{tabularx}{\textwidth}{lrX}
    \toprule
    \tableheadline{Project} & \tableheadline{Age} & \tableheadline{Founder} \\
    \midrule
    Debian        & 1993 & Ian Murdock \\
    Drupal        & 2001 & Dries Buytaert \\
    Fedora        & 2002 & Warren Togami \\
    GNOME         & 1997 & Miguel de Icaza \& Federico Mena Quintero \\
    KDE           & 1996 & Matthias Ettrich \\
    MySQL/MariaDB & 1997 & Michael Widenius \& David Axmark \\
    PHP           & 1994 & Rasmus Lerdorf \\
    Plone         & 1999 & Alexander Limi \& Alan Runyan \\
    PostgreSQL    & 1986 & Lawrence A. Rowe \& Michael R. Stonebraker \\
    Python        & 1989 & Guido van Rossum \\
    \bottomrule
  \end{tabularx}
  \caption[Project Founders and Origins]{Project founders and origins.}
\end{table}

% }}}

\section{Community} % {{{

The previous project analysis has shown, that the communities of each project
seem to be quite different. This will be compared in the following sections.

\subsection{Community Size} % {{{

As all of the analyzed projects share a quite long history, a lot of people
were active in the projects. This of course biases the comparison between
communities, as the actual number of developers is almost always lower than the
total sum of developers in the project history. Also taking in mind that only
the central parts of the projects were analyzed, the resulting number can vary
a lot. For example the GNOME project claims that over 3500 people contributed
to the project. The fact that only GNOME Core was analyzed and no longer active
developers were left out reduces the finding to currently about 350 active
developers of GNOME Core. The same probably holds true for the Drupal project
with about 1200 enlisted developers and for the KDE project with about
1800 active contributors.

Also it is important to carefully distinguish between core developers and other
contributors. Since in some projects only core developers have access to the
code repository, an analysis of code authors is not always a satisfying
approach. The Drupal project for example only has about six to ten distinct
authors to Drupal Core. Only recently they adopted to give credit to the
original author instead of referring to them in the commit message. Another
interesting insight is the Plone project which claims to have about 300 active
core developers, the code repository however only shows a number between
50 to 70 active developers at the same time. The PostgreSQL project on the
other hand gives an exact number of active and retired core developers, only
leaving out the number of minor contributors.

It is not always the case that the actual number of developers is smaller.
Projects like Fedora or Debian with a broad scope and smaller tasks like
package creation can actually have a number of active developers which matches
the claimed number. However one has to keep in mind that the comparison of a
core piece of a project with a full operating system isn't valid and has low
relevance. That said the size comparison must be analyzed carefully with the
above considerations in mind.

% }}}

\subsection{Communication} % {{{

On a worldwide distributed project the communication within a project is one of
the most vital parts. Every project has a wide array of communication methods
in place. Quite every project uses mailing lists as its favourite communication
method. A plausible explanation for this is probably the fact that it is easily
scalable, open and archived.

Although every project has lots of different lists for quite every aspect
within the project, there is always one mailing list which is the most
important. This is true for GNOME with the \emph{desktop-devel}, PHP with the
\emph{internals}, Plone with the \emph{plone-developers}, Python with the
\emph{python-dev}, PostgreSQL with the \emph{pgsql-hackers}, Fedora with the
\emph{devel} and Debian with the \emph{debian-devel} mailing list. In some
projects this single mailing list is split into two, one more general list for
the project and the second for the core project. This is the case for KDE with
the \emph{kde-devel} and \emph{kde-core-devel} mailing lists and MariaDB with
the \emph{maria-developers} and \emph{maria-captains} mailing lists. An
exception is the Drupal project which uses mailing lists however the most
important communication channel is a mailing list akin forum on their project
website.

Additionally in every project there are lots of other direct and indirect
communications such as \ac{IRC} channels, blogs, newsletters and others.

% }}}

\subsection{Conferences} % {{{

As most of the developers are living in different cities, countries or
continents, the communication is almost always handled through the internet.
Therefore conferences and meet-ups provide the possibility to meet in person,
plan and design future goals more easily and to invigorate the community. Each
analyzed project organizes one or more annual conferences in place where
developers and users can meet. It is interesting to note that it only after
years communities started to organize them on a regular basis. It also seems
that most projects existed since years before the first conference was held.
Along with the growth of a project the number of attendees at a conference grew
as well. Another interesting fact is that some projects include the conference
in their release and development schedule, for example the GNOME project in
which the conference provides an extra development boost for the upcoming
release in autumn.

Due to the size of the project and the worldwide distribution of developers,
each project organizes in addition frequent local meet-ups, hackfests, sprints,
smaller conferences or other workshops. These however focus more on
specific parts or goals of the project.

% }}}

\subsection{Roles} % {{{

The analysis of the projects has shown, that all projects use a fairly
hierarchical development structure. Starting from the bottom of the hierarchy
there is always room for simple developers or contributors. Moving one step up,
developers or contributors get more rights but also more responsibilities. For
example one manages certain parts of the project or provides guidance to new
developers. Above these maintainers of a certain subsystem is always a rather
small group managing the project's development. Then, at the top there is a
project leader or a group filling the project leader role. It is also
interesting that some projects split the same function into two roles. Keeping
this fact and the different project scopes in mind, the used structures are
middlingly similar, especially if the projects are working within the same
field.

\begin{figure}[htbp]
  \centering
  \subfloat[Drupal]
    {\includegraphics[width=.35\textwidth]{drupal/roles}} \qquad
  \subfloat[Plone]
    {\includegraphics[width=.35\textwidth]{plone/roles}} \\

  \subfloat[Python]
    {\includegraphics[width=.35\textwidth]{python/roles}} \qquad
  \subfloat[PHP]
    {\includegraphics[width=.35\textwidth]{php/roles}} \\

  \subfloat[GNOME]
    {\includegraphics[width=.35\textwidth]{gnome/roles}} \qquad
  \subfloat[KDE]
    {\includegraphics[width=.35\textwidth]{kde/roles}} \\

  \subfloat[PostgreSQL]
    {\includegraphics[width=.35\textwidth]{postgresql/roles}} \qquad
  \subfloat[MariaDB]
    {\includegraphics[width=.35\textwidth]{mariadb/roles}} \\

  \subfloat[Fedora]
    {\includegraphics[width=.35\textwidth]{fedora/roles}} \qquad
  \subfloat[Debian]
    {\includegraphics[width=.35\textwidth]{debian/roles}} \\
  \caption[Comparison of Development Related Roles]
  {Comparison of development related roles in the projects.}
\end{figure}

Of course this structure varies over time and size of the project. For example
the Debian project which has a very long history in comparison to others has a
filed segmentation with clear duties and responsibilities for each role. A
younger project like MariaDB has a simpler model, which nonetheless has the
described groups on top.

A release team is a critical part of a project, either as a dedicated team or
represented by the project leader. Furthermore the concept of having people who
are responsible for certain subsystems seems to be widely established across
all bigger projects.

It is also interesting to note, that in projects where the project founder is
still active, the structure is more hierarchical with the founder on the top.
This is true for Drupal, PHP and Python. Debian is a special case because the
original project leader passed on his duty, but preserved the project leader
hierarchy. Sometimes the leader gathers other people around him trying to be
not solely responsible for the project, such as in the MariaDB project. In
projects where the leader is no longer involved or does not hold any power,
project leader teams were established. This is true for KDE, Fedora and GNOME.

Even if projects claim to be friendly to newcomers and an open community, the
structures analyzed here show, that one always has to start at the bottom of
the hierarchy, earning trust and providing enhancements to the projects before
being promoted on holding a carrying business. 

% }}}

\subsection{Project Founders} % {{{

The project founders played a vital role during the inception of the project. It
was already explained that even if there are more project founders, often one
single person plays a leadership role. It was also pointed out that some
project founders are still active and lead the project, while others have left
their project.

Therefore the project founders of the analyzed projects can be classified into
three categories. First, project founders which are still active in the project
and have a leadership position. This is the case for the Drupal project, as
well as for Python and PHP. In all of those projects, the development structure
might look different without the leader. They define goals, releases, new
features and are often the decision maker. The opposite are projects where the
original founder no longer is present. Fedora, Debian and PostgreSQL are good
examples here. All named projects established a leadership group which defines
the future of the project. Debian still has a project leader, who gets voted by
the community. At last there are projects, where the original leader is still
active and sometimes gives input but has no power. In the KDE and GNOME
projects the founders are working on related projects or topics and sometimes
give their opinion, but they can't decide anything. In the Plone and MariaDB
project the founders are members of the leadership team but have no special
role. At most they can give their opinion and community members might give
their notion a higher value.

% }}}

% }}}

\section{Release Process} % {{{

The release process seems to be the most diverse item in this analysis, as most
projects have gone through different processes and workflows during their time
of existence. Nevertheless some of the different stages of the projects seem to
be quite equivalent to each other.

\subsection{Versioning Scheme} % {{{

All projects, except Drupal and Fedora, use a three digit version naming
scheme. It consists of a major, minor and micro number, which characterizes a
release. The Drupal project uses a two digit versioning scheme, which defines
major and minor releases. It is important to note, that the minor releases of a
major release only include bug fixes and therefore are following the scheme of
a major and micro version number. The Fedora Project just uses a single number
to represent major releases. Due to the structure and automatic update process,
bug fixes can be included automatically in a running system. For comparison the
Debian project also pushes bug fixes to the users, however it additionally
releases those changes as further bug fix releases.

It is true for all projects, that an increment in the major number defines
often a backwards incompatible release with new features. Such a major release
is often a ground breaking change and often defines a new era for the project.
A major releases is generally extremely scarce. This is also reflected by the
comparatively low major number the projects have, for example GNOME~3, KDE~4,
Plone~4, Drupal~7 and so on. Fedora is an exception here, as every new release
increments the major number by one, which currently is 16.

The minor number almost always defines a feature release which is backwards
compatible to the previous major release. There are some exceptions where this
is not the case, for example in the PHP 5.x release cycle. However those
changes can be seen as minor when comparing them to the incompatibilities of
two major releases. Also, if such changes occur, most projects provide an
upgrade path. Except Drupal and Fedora, all projects use such a number.

In some projects, an increment of the minor version number defines a major
release. In other words, regardless of whether the major or minor number
changes the release is always a major release. This is true for GNOME, PHP and
Debian.

The GNOME project always increments its minor number by two to define a stable
release cycle. An odd minor number therefore always defines development
releases.

\begin{table}[htb]
  \centering
  \begin{tabularx}{\textwidth}{Xccc}
    \toprule
    \tableheadline{Project} & \tableheadline{Major} & \tableheadline{Minor} & \tableheadline{Micro} \\
    \midrule
    Debian          & \checkmark & \checkmark & \checkmark \\
    Drupal          & \checkmark &            & \checkmark \\
    Fedora          & \checkmark &            &            \\
    GNOME           & \checkmark & \checkmark & \checkmark \\
    KDE             & \checkmark & \checkmark & \checkmark \\
    MySQL/MariaDB   & \checkmark & \checkmark & \checkmark \\
    PHP             & \checkmark & \checkmark & \checkmark \\
    Plone           & \checkmark & \checkmark & \checkmark \\
    PostgreSQL      & \checkmark & \checkmark & \checkmark \\
    Python          & \checkmark & \checkmark & \checkmark \\
    \bottomrule
  \end{tabularx}
  \caption[Used Versioning Schemes in the Analyzed Projects]{Used versioning
    schemes in the analyzed projects.}
\end{table}

Micro releases are a number of subset releases for the above mentioned feature
releases. They are only allowed to include bug and security fixes, no new
features and no incompatibilities. Except Fedora, all projects use such a
number.

In a nutshell it can be said that all projects use a similar versioning scheme
with some minor exceptions, which do not have an all to big impact.

% }}}

\subsection{Release Schedule} % {{{

The release schedule leads to new major releases and often includes minor and
bug fix releases. They are therefore a vital point for each project's future. As
such most projects have established a rather detailed release schedule which
they accurately try to follow. Nevertheless there are some differences while
comparing each project's schedule which will be outlined in the following.

The most distinguished difference is the decision of a fixed release cycle. A
fixed release cycle iterates over a given time frame and provides expectable
release dates. Projects like Plone, GNOME and KDE projects use a fixed release
cycle with a duration of six months. It means, that each year there will be two
major releases including a number of minor or bug fix releases. Also Fedora uses
a similar cycle, as it is more or less bound to the GNOME release cycle. The
PostgreSQL and PHP projects use a fixed release cycle with a duration of one
year.

Although projects like Drupal and Python do not use a fixed release cycle for
major releases, they do however use a fixed release cycle for minor and bug fix
releases. The Python project publishes minor releases roughly every 18 months
and bug fix releases roughly every six months. The Drupal project publishes
bug fix releases on a monthly basis.

Only MySQL/MariaDB and Debian do not have a fixed release cycle. At least the
Debian project is currently trying to provide a fixed release schedule.

All projects however have some sort of freezes as part of the schedule, such as
code, features, translatable strings, documentation and more. Freezes should
help to confine areas where new bugs or drawbacks could appear. Of course they
are restricted to new development and developers can still apply bug fixes.
However they will be analyzed deeply before they can be applied. As such
freezes are a vital part of all analyzed project schedules. Also, they often
keep in effect for the released major branch. One of the most obvious
consequences is the fact that bug fix releases only include fixes and no new
features or user interface changes.

\begin{figure}[htbp]
  \centering
  \subfloat[Drupal]
    {\includegraphics[width=.46\textwidth]{drupal/cycle}} \qquad
  \subfloat[Plone]
    {\includegraphics[width=.46\textwidth]{plone/cycle}} \\

  \subfloat[Python]
    {\includegraphics[width=.46\textwidth]{python/cycle}} \qquad
  \subfloat[PHP]
    {\includegraphics[width=.46\textwidth]{php/cycle}} \\

  \subfloat[GNOME]
    {\includegraphics[width=.46\textwidth]{gnome/cycle}} \qquad
  \subfloat[KDE]
    {\includegraphics[width=.46\textwidth]{kde/cycle}} \\

  \subfloat[PostgreSQL]
    {\includegraphics[width=.46\textwidth]{postgresql/cycle}} \qquad
  \subfloat[MariaDB]
    {\includegraphics[width=.46\textwidth]{mariadb/cycle}} \\

  \subfloat[Fedora]
    {\includegraphics[width=.46\textwidth]{fedora/cycle}} \qquad
  \subfloat[Debian]
    {\includegraphics[width=.46\textwidth]{debian/cycle}} \\
  \caption[Representation of Development Cycles]
  {Schematic representation of major and minor development cycles of the
    analyzed projects.}
\end{figure}

The most interesting fact are when the freezes occur and with which
restrictions. The GNOME project as well as the Debian project have established
a freeze shortly before their final release including code, feature, user
interfaces, \ac{API} and string freezes. The freeze period of the GNOME project
however has a defined start and end date, since the release schedule is a fixed
one. The KDE project on the other hand distributed the different freezes more
evenly throughout the release cycle. It also differentiates between soft and
hard freezes. The freezes are set in order and always have a soft freeze before
a hard freeze. This applies for features, strings and \ac{API} freezes. The
Plone project is set somewhat in the middle having a feature freeze two months
before the final release and then applying more and more restrictions during
the releases of alpha, beta and release candidate versions. The Fedora project
is quite similar to that approach, however has its feature freeze roughly three
months before the final release. Following an alpha and beta release more
restrictions apply.

The Python project only allows new features until the first beta release which
occurs about two months before the final release. After that point only bug
fixes can be applied.

The PHP project has changed the layout of its schedule a lot during the last
years while trying to elaborate its yearly release schedule. Currently the
feature freeze is in act about eight months before the final release, confining
other freezes more and more until the first alpha release. Further beta
releases from that point until the final releases do only get bug fixes.

The not fixed release schedule of MySQL/MariaDB forces the project to establish
criteria when specific releases can happen. A such criteria for a feature
freeze is the publication of alpha releases. After that no new features are
allowed and only bugs get fixed. This continues until the final release
is considered to be bug free.

Like MySQL/MariaDB the Drupal project does not have a fixed release cycle and
therefore arranges the development and freezes into different phases. Each
phase comes with its freezes in the following order code, \ac{API}, feature,
user interface and string freeze. After all freezes have been declared a number
of alpha, beta, release candidate and lastly the final release is published.

The PostgreSQL has replaced a single feature freeze with a series of Commit
Fest which were explained earlier. In fact it is the only analyzed project
which does not have a single freeze but a series of development milestones
which limit new features and development time by time.

It is also interesting to note, that every project uses some kind of alpha,
beta and release candidate versions before publishing the final release. The
naming is of course different. For example the GNOME project does not provide a
special naming, except for the odd minor number, while other projects, such as
Drupal or Debian name their release accordingly.

% }}}

% }}}

\section{Development} % {{{

While the release cycle defines the process around new releases, the
development process of a project defines the actual shape of it. As such it is
vital to understand how projects are being lead, structured and divided.

\subsection{Development Lead} % {{{

The development of a project is always lead by a single person or a group of
people. All analyzed projects' development had a form of leadership. Each
project showed a hierarchical structure when it came to decisions related to
the development of a project.

Often the development is lead by the founder, if they are still active. In the
Python, Drupal and PHP projects the original founder has a great influence when
it comes to decide on changes or new features. Projects like Fedora, MariaDB,
Plone and PostgreSQL have established leadership groups who define the
direction of the project as well as decide over new features. While the Fedora
and Plone project have an arbitrary or elected group they do not necessarily
provide enhancements themselves. The MariaDB and PostgreSQL projects on the
other hand are mostly enhanced and lead by the same group.

While the KDE and GNOME projects have established core or release teams they do
not necessarily have power on areas other than their duties. Also due to their
modular structure, the concrete development is mostly lead by the maintainers
of single areas. A minor exception is GNOME's design team which is involved in
the shaping of user interfaces. Since the team cannot dictate other GNOME
members what to do, the above assumption still holds.

The Debian project is structured very hierarchically and as such the overall
direction is set by the elected project leader. The technical committee then
looks after the concrete implementation while the single team leaders are in
place for certain areas of the project.

% }}}

\subsection{Development Workflow} % {{{

The development workflow of a project defines how changes and enhancements are
brought into the project. Some of the analyzed projects have established a
feature inclusion process which will be highlighted in the next section.

In all analyzed projects it is possible to commit small patches and
enhancements directly to the project's repository or let it be committed by a
core contributor if one has not enough rights. Such changes are not tied to an
existent feature inclusion process. As such this often only comprises bug fixes
and other changes similar in size. This is often assisted by a bug tracking
system which each of the analyzed project uses.

It seems that all changes committed to the repositories are reviewed by other
members of the project. In case of not gaining acceptance the patch is often
discussed in the project's mailing lists or other suitable means of
communication. It is also possible that patches are sent to the mailing list or
to a bug tracking system and discussed there before being applied. Quite all
projects have some sort of team leaders or maintainers who can decide about a
certain patch for a certain subsystem of the project.

% }}}

\subsection{Feature Inclusion Process} % {{{

The feature inclusion process defines, if established, how and when features or
big changes come into a project. In the analyzed projects a quite diverse
passel of processes has been found. It ranges from a very structured and well
documented process like the one in the Python project to a more dynamic
approach like the one in the KDE project.

The most structured and well laid-out process is Python's \ac{PEP} process. It
defines a well documented approach of new features and big changes from its
inception until either acceptance or withdrawal. This process seems to work
well for the Python community as well for development, community and leadership
processes.

\begin{figure}[ph]
  \centering
  \subfloat[Drupal]
    {\includegraphics[width=.467\textwidth]{drupal/feature}} \qquad
  \subfloat[Plone]
    {\includegraphics[width=.467\textwidth]{plone/feature}} \\

  \vspace{0.5cm}

  \subfloat[Python]
    {\includegraphics[width=.467\textwidth]{python/feature}} \qquad
  \subfloat[PHP]
    {\includegraphics[width=.467\textwidth]{php/feature}} \\

  \vspace{0.5cm}

  \subfloat[GNOME]
    {\includegraphics[width=.467\textwidth]{gnome/feature}} \qquad
  \subfloat[KDE]
    {\includegraphics[width=.467\textwidth]{kde/feature}} \\

  \vspace{0.5cm}

  \subfloat[PostgreSQL]
    {\includegraphics[width=.467\textwidth]{postgresql/feature}} \qquad
  \subfloat[MariaDB]
    {\includegraphics[width=.467\textwidth]{mariadb/feature}} \\

  \vspace{0.5cm}

  \subfloat[Fedora]
    {\includegraphics[width=.467\textwidth]{fedora/feature}} \qquad
  \subfloat[Debian]
    {\includegraphics[width=.467\textwidth]{debian/feature}} \\

  \vspace{0.5cm}

  \caption[Representation of Feature Inclusion Processes]
  {Schematic and simplified representation of feature inclusion processes of
    the analyzed projects.}
\end{figure}

Interestingly other projects started to adopt this process with minor changes.
For example the \ac{PLIP} process, which is used by the Plone project only uses
it for development enhancements and not for community issues. As such the
process is a simpler and more focused to development compared to the \ac{PEP}
process.

Also the PHP project adopted a version of the \ac{PEP} process which is called
\ac{RFC}. Since it also focuses on development, it is similar to the \ac{PLIP}
process. A minor difference is the separate voting step on each proposal, which
is not existent in the other two.

A slightly more dynamic approach is the Drupal initiative process where each Drupal
developer can propose an initiative with a certain goal. At a certain point in
time the Drupal project founder and leader Dries Buytaert chooses the
initiatives the project should follow for the next release.

Similar to this, but slightly more structured and not backed by a single person
is the feature based development process of GNOME and Fedora. In both projects,
features can be proposed which are then chosen by the GNOME release team and
the \ac{FESCo}. In both projects proposals follow a certain format and are more
structured when compared to Drupal initiatives. Also, in both projects the
named groups have the possibility to drop or postpone a certain feature if it
is not ready for the upcoming release.

The Debian project uses goals which can be proposed by community members and
are chosen by the release team. This is antithetic to the previous projects,
since goals are just single wordings without any structure. No concrete steps,
involved parties, code or other are trailed to the goal.

The KDE and MariaDB projects use an even more dynamic approach where members
can propose features, but they have to be picked up and implemented by
developers. As such there is no real instance which could prohibit the
development of a certain feature.

The PostgreSQL project is the only project which has removed such a specific
inclusion process altogether and replaced it with a series of Commit Fests. A
feature can be brought into a Commit Fest and evolves over time until it can be
committed. Given the analysis such a series of feature reviews is equivalent to
a series of feature inclusion processes where the goal is always to get a
certain enhancement into a project instead of discarding it directly.

% }}}

% }}}

% }}}
