\chapter{Theoretical Background} % {{{
\label{chap:theoretical background}

In order to analyze and compare development workflows a series of scientific
research methodologies were used. This chapter gives an introduction to those
and tries to vindicate their usage.

\section{Grounded Theory} % {{{

\begin{figure}[htbp]
  \centering
  \includegraphics[width=0.7\textwidth]{grounded_theory}
  \caption{Process of theory building using the grounded theory as research
    method according to \citeauthor{Strauss1990} \cite{Pandit1996}.}
\end{figure}

The \emph{grounded theory} approach is a systematic research methodology first
published by \textcite{Glaser1967} in \citeyear{Glaser1967}. It first was
applied in social sciences gathering data and discovering a theory after an
analysis of the rallied data. Grounded theory is mostly used in qualitative
research and operates quite different from traditional research methods. It
always starts with the collection of data about the researched subject.
Thereafter the collection items can be marked with codes which later can be
grouped into sort of categories. They are the base of an emerging theory which
can be established out of the gathered data and it's analysis.

However \citeauthor{Glaser1967} have established different opinions on the
grounded theory approach \cite{Heath2004}. This has lead to different
appendages, which either favorite the Glaser or the Strauss understanding of
how the grounded theory works. The most important difference seems to be the
technique used to code the samples into categories. Glaser favors a
systematical and well defined approach for the coding process while Strauss
favors a more dynamic approach coding the elements as soon as they become
visible. The second approach was described by \textcite{Strauss1990} in
\citeyear{Strauss1990}. Both methods have their benefits and downsides, however
in this analysis the \citeauthor{Strauss1990} approach was chosen as a better
fitting research method.

Due to the quite dynamic development approach and nescience about the different
development processes, a grounded theory approach was reviewed as a good
scientific research method to gather all the required data and use it as a base
research method.

% }}}

\section{Qualitative Content Analysis} % {{{

\begin{figure}[htbp]
  \centering
  \includegraphics[width=0.7\textwidth]{mayring}
  \caption{Step model of inductive category development in quantitative content
    analysis according to \citeauthor{Mayring2008} \cite{Mayring2000,Mayring2008}.}
\end{figure}

Another quite useful research method is the \emph{qualitative content analysis}
first published by \textcite{Mayring2008}. It is quite related to the previous
mentioned grounded theory however differs from it's approach. The main idea is
to apply methods of qualitative content analysis to the research question and
transfer them to a step by step interpretation which can be used for
quantitative content analysis \cite{Mayring2000}. This leads to a framework one
can use to provide an empirical yet methodologically controlled analysis which
follows a previously established content analysis canon.

The centerpiece of this method is the finding and establishment of categories
in which the analyzed data can be put. It also considers ways to carefully
check and review the found categories. This is also known as \emph{feedback
loops}. The establishment of categories can be done either in an inductive or
deductive way. For this analysis it makes of course sense to use an inductive
category development.

When using an inductive approach for category generation, the research method
begins with the research question and the ensuing determination of category
definition. This defines what aspects will be taken into account during the
analysis. Following the analysis further possible categories may appear. Those
categories will then be revised and checked if they are valid categories for
this analysis. Furthermore they can be split up or be set as subcategories of
previously found categories. This step will be iterated through the research
data until a final interpretation and analysis can be established.

This qualitative content analysis approach supplements the grounded theory in a
quite useful way, especially for the comparison of different development
models. As such this research method was taken into account.

% }}}

\section{Application in Computer Science} % {{{

Qualitative research methods such as the above have their origin in the social
sciences. However they also have their right of existence in computer science.
They provide excellent research methods and tools to examine existing phenomena
and domains but not only. As such they are especially useful in areas like
software engineering, computer science education or computer science research
methods. The following list should stand for an array of examples from
different areas of the computer science research field.

\textcite{Hazzan2006} for example discuss the use and plausibility of
qualitative research methods in computer science education. Similarly
\textcite{Meerbaum-Salant2010} explore different ways how to learn computer
science concepts with qualitative analysis methods.

\textcite{Armstrong2006} uses qualitative research methods to discuss the usage
of object oriented programming vocabulary in different papers and books. The
grounded theory is used by \textcite{Perry2000} to outline the strengths and
weaknesses of empirical research in software engineering. It is also used by
\textcite{Sarker2001} for building and developing a process model of
collaboration in virtual teams.

Finally \textcite{Bainbridge2003} use the grounded theory as well to analyze
music queries. \textcite{Kaplan1994} use qualitative research methods for
evaluating computer information systems.

As one can see, there is quite some effort going on to use qualitative research
methods, especially the grounded theory, in computer science. However it must
be noted, that such a qualitative analysis might not fit for every research
matter and often is suited for meta or process analysis.

% }}}

\section{Software Engineering Comparison Models} % {{{

Research in software engineering has brought up many different software
engineering models which are suitable for different types of environments. For
a later comparison the most renowned and appositely models should be presented
in the following.

\subsection{Traditional Software Engineering Models} % {{{

There exist many different traditional software engineering models. The in the
following described \emph{waterfall} and \emph{spiral model} should stand as an
example and will be used for the comparison later on.

\begin{figure}[htbp]
  \centering
  \includegraphics[width=0.6\textwidth]{waterfall}
  \caption{The waterfall software engineering model according to \textcite{Royce1970}}
\end{figure}

The first formal description of the waterfall model was published by
\textcite{Royce1970} in \citeyear{Royce1970}. It is interesting to note that
\citeauthor{Royce1970} did not use the term \emph{waterfall} to describe his
model. Furthermore he described it as a model which is not appropriate for
software engineering. However it is still very popular and widely known.

The model follows a sequential process which flows from top to the bottom. That
is also the reason of it's name. The original sequential phases are from top to
bottom System Requirements, Software Requirements, Analysis, Program Design,
Coding, Testing and Operations. However there exist also several modified
waterfall models which change the single sequences and domains.

\begin{figure}[htbp]
  \centering
  \includegraphics[width=0.8\textwidth]{spiral}
  \caption{The spiral software engineering model according to \textcite{Boehm1988}}
\end{figure}

\textcite{Basili1975} were then the first description of iterative enhancements
inside a software engineering life cycle. In \citeyear{Boehm1988},
\textcite{Boehm1988} created the spiral model as a response to the waterfall
model. The spiral model can already be described as an evolutionary or
iterative development model. As the name suggest it is devised as a spiral
including steps like risk management. It therefore combines the systematic
approach by the waterfall model with an evolutionary development approach which
allows to incrementally enhance the product.

A typical cycle of the spiral always starts with the specification of goals and
constraints of this cycle. It then proceeds to risk management which tries to
identify risks or to provide alternatives. Next the defined goals will be
developed and tested. Lastly the next cycle is planned.

% }}}

\subsection{Agile Software Engineering Models} % {{{

Agile software engineering models base in comparison to traditional software
engineering models on incremental and iterative development which are known to
be quick and flexible to new requirements or changes \cite{Beck1999}. The
\emph{\acl{XP}} and \emph{Scrum} development models should stand as an example
for this area.

\begin{figure}[htbp]
  \centering
  \includegraphics[width=\textwidth]{xp}
  \caption{The \acl{XP} software engineering model according to \textcite{Beck1999a}}
\end{figure}

\ac{XP} is a software development model first published by
\citeauthor{Beck1999} \cite{Beck1999a} in \citeyear{Beck1999}.
\citeauthor{Beck1999} set his goal to improve existing software engineering
models by improving the overall software quality of a product and the
responsiveness of changes required by customers or other incorporated parties.
The main difference to existing models are frequent releases and in general
short development cycles. It also includes other paradigms such as unit tests,
flat management, simplicity, awareness of user requirement changes and frequent
communication with the customer. It has to be noted, that \citeauthor{Beck1999}
does not promote \acl{XP} as a finished concept, however it should be more of a
process existing teams can adapt to \cite{Beck1999}.

\begin{figure}[htbp]
  \centering
  \includegraphics[width=\textwidth]{scrum}
  \caption{The Scrum software engineering model according to \textcite{Schwaber1995}}
\end{figure}

The Scrum software development model was first referred by
\textcite{DeGrace1990} in \citeyear{DeGrace1990}. Not until
\citeyear{Sutherland1995} however the Scrum development model was presented
formally by \textcite{Sutherland1995} and \textcite{Schwaber1995}. As the
\acl{XP} development model it is an incremental and iterative development
model. The Scrum development model usually contains three major roles who
ensure the process model is followed, represent the customer and represent the
development team. The basic idea behind Scrum are so called \emph{sprints}
which last between a week and one month. These sprints are planned before they
are executed and represent one development cycle in which parts of the project
will be developed. Another important part of the model is the backlog which
keeps track of which features are to be done in which priority. These tools
should ensure the frugalness of new requirements by the customer or other
constraints.

% }}}

% }}}

% }}}
