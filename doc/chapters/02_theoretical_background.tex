\chapter{Theoretical Background} % {{{

In order to analyze and compare development workflows a series of scientific
research methodologies were used. This chapter gives an introduction to those
and tries to vindicate their usage.

\section{Grounded Theory} % {{{

The grounded theory approach is a systematic methodology first published by
\textcite{Glaser1967} in \citeyear{Glaser1967}. It first was applied in social
sciences gathering data and discovering a theory after an analysis of the
rallied data. Grounded theory is mostly used in qualitative research and
operates quite different from traditional research methods. It always starts
with the collection of data about the researched subject. Thereafter the
collection items can be marked with codes which later can be grouped into sort
of categories. They are the base of an emerging theory which can be established
out of the gathered data and it's analysis.

However \citeauthor{Glaser1967} have established different opinions on the
grounded theory approach \cite{Heath2004}. This has lead to different
appendages, which either favorite the Glaser or the Strauss understanding of
how the grounded theory works. The most important difference seems to be the
technique used to code the samples into categories. Glaser favors a
systematical and well defined approach for the coding process while Strauss
favors a more dynamic approach coding the elements as soon as they become
visible. The second approach was described by \textcite{Strauss1990} in
\citeyear{Strauss1990}. Both methods have their benefits and downsides, however
in this analysis the \citeauthor{Strauss1990} approach was chosen as a better
fitting research method.

Due to the quite dynamic development approach and nescience about the different
development processes, a grounded theory approach was reviewed as a good
scientific research method to gather all the required data and use it as a base
research method.

% }}}

\section{Qualitative Content Analysis} % {{{

\begin{figure}[htbp]
  \centering
  \includegraphics[width=0.7\textwidth]{mayring}
  \caption{Step model of inductive category development in quantitative content
    analysis according to \citeauthor{Mayring2008} \cite{Mayring2000,Mayring2008}.}
\end{figure}

Another quite useful is a qualitative content analysis first published by
\textcite{Mayring2008}. It is quite related to the previous mentioned grounded
theory however differs from it's approach. The main idea is to apply methods of
qualitative content analysis to the research question and transfer them to a
step by step interpretation which can be used for quantitative content analysis
\cite{Mayring2000}. This leads to a framework one can use to provide an
empirical yet methodologically controlled analysis which follows a previously
established content analysis canon.

The centerpiece of this method is the finding and establishment of categories
in which the analysed data can be put. It also considers ways to carefully
check and review the found categories. This is also known as \emph{feedback
loops}. The establishment of categories can be done either in an inductive or
deductive way. For this analysis it makes of course sense to use an inductive
category development.

When using an inductive approach for category generation, the research method
begins with the research question and the ensuing determination of category
definition. This defines what aspects will be taken into account during the
analysis. Following the analysis further possible categories may appear. Those
categories will then be revised and checked if they are valid categories for
this analysis. Furthermore they can be split up or be set as subcategories of
previously found categories. This step will be iterated through the research
data until a final interpretation and analysis can be established.

This qualitative content analysis approach supplements the grounded theory in a
quite useful way, especially for the comparison of different development
models. As such this research method was taken into account.

% }}}

\section{Application in CS/CSE} % {{{

% }}}

\section{Software Engineering Comparison Models} % {{{


% }}}

% }}}
