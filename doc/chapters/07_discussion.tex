\chapter{Discussion} % {{{
\label{chap:discussion}

With the analysis performed and the single findings compared, this chapter
offers a discussion and comparison with related findings and the presented
software engineering methods. Finally the main research results are outlined.

The probably most obvious common ground in the analysis was the establishment
of a project analysis catalogue, with which \ac{FOSS} projects could be
analyzed systematically and satisfactorily for the goals of this thesis. While
the previous chapter described the differences between the projects, they still
share general mutuality of structure and processes. So to say the projects are
similar with equal structures and processes when viewed in a simplified manner.

Another important issue to note is that although it was not examined in this
analysis, all projects seem to evolve their structure and processes over time.
This was shown by \textcite{Scacchi2006}, \textcite{Godfrey2000} but also by
\textcite{Johnson2001} who claims that all \ac{FOSS} projects start as a closed
prototype and evolve into an open project over time.

\section{Project Origin} % {{{

As shown before, most projects are initiated to solve a personal issue or
problem. This fact has also been analyzed in related studies, such as
\textcite{Raymond1998}, \textcite{Lakhani2003}, \textcite{Hertel2003} or
\textcite{Johnson2001}. In some cases the projects develop from existing
projects or scientific studies, such as the Python project. In other the
project is launched by a company with the primary motivation of economic
success.

The analysis has shown, that despite the age of a project, many still grow at
least linearly. Some projects seem to maintain stability over time, while
others, such as the PHP project, seem to decline. Similar results have been
shown by \textcite{Godfrey2000}, \textcite{Roets2007} or \textcite{Ogawa2007}.
\textcite{Schweik2003} identified a life cycle with three phases: project
initiation, going open and project growth, stability or decline. This
definition lines up very well with the findings of this analysis.

% }}}

\section{Community} % {{{

In contrast to the Bazaar model by \textcite{Raymond1998} the project structure
in the analyzed projects was always hierarchical, even if there was only a flat
hierarchical structure. This affirms the findings by \textcite{Crowston2005}.
Also \textcite{Bezroukov1999,Bezroukov1999a} criticizes the Bazaar model as too
simplistic and not matching. As noted earlier, \textcite{Ghosh2005} analyzed
several different models ranging from hierarchical to completely flat, Bazaar
like structures. The latter was not identified in any of the analyzed projects
and even if they did differ in terms of hierarchy levels, the general
hierarchical structure was similar. That of course only is proven for the
analyzed projects and not for all \ac{FOSS} projects.

This repeats itself when one analyzes the project structure with a a leader or
leadership group who drive the project. This correlates with the findings of
\textcite{Johnson2001}, \textcite{Crowston2005a}, \textcite{Warsta2003} and
\textcite{Krishnamurthy2002}.

While the analyzed projects do have a hierarchical structure, the communities
are nevertheless very welcoming and it seems to be relatively easy to enter a
community and step up the ladder. In this aspect the analyzed models resemble
the Bazaar model, which claims that it is very easy for new developers to join
the project and take charge of important parts of the project.

\begin{figure}[htbp]
  \centering
  \includegraphics[width=0.9\textwidth]{structure}
  \caption[\acl{FOSS} project development structure]
  {\acl{FOSS} project development structure as proposed by \textcite{Crowston2005}.}
\end{figure}

Regarding the communication within the projects, the mailing list is the most
used communication method in \ac{FOSS} projects as shown by
\textcite{Schweik2003}, \textcite{Ogawa2007} and \textcite{Kim2003}. This
finding was considered to be true, as all projects do have mailing lists and in
fact the most important communication channel seems to be the mailing list.
Most projects do have several mailing lists in place with one or two being the
most important ones where the development of a project is planned.

% }}}

\section{Release Process} % {{{

An interesting fact is the move towards fixed release cycles. Most of the
analyzed projects adapted a fixed cycle and others are in the process of
transition. Fixed cycles can also be found in agile software engineering
methods, such as \acl{XP} or the Scrum method. While not equal to those
methods, the analyzed projects seem to benefit from such an approach. The
release schedule with fixed cycles is often already fixated before development
starts and each project tries to stick to it as close as possible. Development
releases and freezes are quite similar in all projects. The only main
difference is the number of development releases and freezes and the point in
time when freezes occur and what part of the project they do cover.

As shown by \textcite{Mockus2002} the role of the release manager is vital for
every project. The analysis reflects this where each project has either a
single person or a team in place to manage releases, freezes and all other
related duties. Also \textcite{Crowston2005} rates the release managers as one
of the most important people next to the founder or leader. In some cases, this
role is also the same, especially in new projects.

% }}}

\section{Development} % {{{

While it certainly is not used in mature projects, the waterfall model by
\textcite{Royce1970} often appears in new projects or prototypes. It is also
interesting to note, that the phases described in the waterfall model are often
found in development processes of \ac{FOSS} projects as \textcite{Roets2007}
point out. Iterative or evolutionary software engineering models however are
the most common pattern, especially in mature projects. The spiral model by
\textcite{Boehm1988} is a good example and gives a good first match when
compared to the iterative releases and development schedules.
\textcite{Roets2007} however make a strong case for the spiral model not being
used or found in \ac{FOSS} projects.

\begin{figure}[htbp]
  \centering
  \includegraphics[width=\textwidth]{ossd}
  \caption[Life cycle model of \acl{FOSS} projects]
  {Life cycle model of \acl{FOSS} projects as proposed by \textcite{Roets2007}.}
\end{figure}

Agile methods seem to be much more appropriate when comparing them with
\ac{FOSS} development processes. The iterative process, the close relationship
with the user and quick development and prototyping seem to make an argument
for agile methods. However \textcite{Koch2004}, \textcite{Warsta2003} proof
differences between agile methods and \ac{FOSS} development methods. They point
out an array of differences, most importantly the non available co-location of
developers, in which the processes differ. However they agree that the
processes show clear similarities. For example the development workflow of the
PostgreSQL with its regular commit fests resembles important steps in the Scrum
software engineering method, in the Scrum terminology named sprints. Also
\textcite{Roets2007} agree that no existing software engineering model
resembles the development methods of \ac{FOSS} projects accurately. They
admittedly come up with a proposal of a \ac{FOSS} development cycle which
closely matches the given results. This proposal is based on the findings by
\textcite{Jorgensen2001}.

% }}}

% }}}
