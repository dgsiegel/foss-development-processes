\chapter{Discussion} % {{{
\label{chap:discussion}

With the analysis done and the single findings compared, this chapter offers a
discussion and comparison with related findings and the presented software
engineering methods. Finally the main research results are outlined.

The probably most obvious common ground in the analysis was the establishment
of a project analysis catalogue, with which \ac{FOSS} projects could be
analyzed satisfactorily for the goals of this thesis. While the previous
chapter of course described the differences between the project, they still
share general mutuality of structure and processes. So to say the projects are
quite similar with equal structures and processes when viewed in a simplified
manner.

Another important issue to note is that although it was not examined in this
analysis, all project seem to evolve their structure and processes over time.
This was shown by \textcite{Scacchi2006}, \textcite{Godfrey2000} but also by
\textcite{Johnson2001} who claims that all \ac{FOSS} projects start as a closed
prototype and evolve into an open project over time.

\section{Project Origin} % {{{

As shown before, most projects get started to solve a personal issue or
problem. This can be also found in related studies, such as shown by
\textcite{Raymond1998}, \textcite{Lakhani2003}, \cite{Hertel2003} or
\textcite{Johnson2001}. In some cases the projects arise out of existing
projects or scientific studies, such as the Python project shows. In other the
project is developed from a company, in which the primary motivation is
economic success.

The analysis has shown, that despite the age of a project, many still grow at
least linearly. Similar results have been shown by \textcite{Godfrey2000},
\textcite{Roets2007} or \textcite{Ogawa2007}. \textcite{Schweik2003} propose a
life cycle with three phases: project initiation, going open and project
growth, stability or decline. This lines up very well with the findings of this
analysis.

% }}}

\section{Community} % {{{

In contrast to the \emph{Bazaar} model by \textcite{Raymond1998} the project
structure in the analyzed projects was always hierarchic, even if there only
was a flat hierarchic structure. This accords with the findings by
\textcite{Crowston2005}. As noted earlier, \textcite{Ghosh2005} proposed
several different models ranging from hierarchic to completely flat,
bazaar-like structures. This however was not the case in the analyzed projects
and even if they did differ in terms of hierarchy levels, the general
hierarchic structure was similar. That of course only holds for the analyzed
projects and not for all \ac{FOSS} projects.

This also reflects when one looks into the structure of the project in which
every project has a leader or leadership group who drive the project. This
correlates with the findings of \textcite{Johnson2001},
\textcite{Crowston2005a}, \textcite{Warsta2003} and
\textcite{Krishnamurthy2002}.

\begin{figure}[htbp]
  \centering
  \includegraphics[width=0.85\textwidth]{structure}
  \caption{\ac{FOSS} project development structure as proposed by \textcite{Crowston2005}}
\end{figure}

As shown by \textcite{Schweik2003}, \textcite{Ogawa2007} and
\textcite{Kim2003} the mailing list is the most used communication
method in \ac{FOSS} projects. In the analyzed projects this finding was
considered to be true, as all projects do have mailing lists and the most
important communication channel seems to be in fact the mailing list.

% }}}

\section{Release Process} % {{{

An interesting fact seems to be the move towards fixed release cycles. Most of
the analyzed projects adapted a fixed cycle and others are in the process of
the transition. Fixed cycles can also be found in agile software engineering
methods, such as \ac{XP} or the \emph{Scrum} method. While not equal to those
methods, the analyzed projects seem to see benefits in such an approach. Coming
with fixed cycles also the release schedule is often already fixated before
development starts and each project tries to stick to it as close as possible.
Development releases and freezes seem to be quite similar in all projects. The
only main difference is the number of development releases and freezes and the
point in time when freezes occur and what they do cover.

As shown by \textcite{Mockus2002} the role of the release manager is a vital
for every project. That reflects in the analysis where each project has either
a single person or a team in place to manage releases, freezes and all other
duties of releases. Also \textcite{Crowston2005} rates the release managers as
one of the most important people next to the founder or leader. In some cases,
this role is also the same, especially in young projects.

% }}}

\section{Development} % {{{

While it certainly is not used in mature projects, the \emph{waterfall} model
by \textcite{Royce1970} can often been seen in new projects or prototypes. It
is also interesting to note, that the phases described in the \emph{waterfall}
model are often found in development processes of \ac{FOSS} projects as
\textcite{Roets2007} point out. Iterative or evolutionary software engineering
models however are most often found, especially in mature projects. The
\emph{spiral} model by \textcite{Boehm1988} is a good example and gives a good
first match when compared to the iterative releases and development schedules.
\textcite{Roets2007} however make a strong case for the \emph{spiral} model not
being used or found in \ac{FOSS} projects.

Agile methods seem to be much more appropriate when comparing them with
\ac{FOSS} development processes. The iterative process, the close relationship
with the user and quick development and prototyping seem to make an argument
for agile methods. However \textcite{Warsta2003} and \textcite{Koch2004} show
differences between agile methods and \ac{FOSS} development methods. They show
an array of differences, most importantly the not available co-location of
developers, in which the processes differ. However, they agree that the
processes show similarities. For example the development workflow of the
PostgreSQL with it's regular commit fests pretty much resembles important steps
in the \emph{Scrum} software engineering method, in the \emph{Scrum}
terminology named sprints. Also \textcite{Roets2007} agree that no software
engineering model resembles the development methods of \ac{FOSS} projects
accurately.

% }}}

% }}}
